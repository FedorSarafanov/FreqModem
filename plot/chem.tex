% Observer/Estimator
% Author: Dominik Haumann
\documentclass[tikz]{standalone}

\usepackage[english,russian]{babel}
% \usepackage[defaultsans]{droidsans}
% \renewcommand*\familydefault{\sfdefault} %% Only if the base font of the document is to be typewriter style
\usepackage[T2A,T1]{fontenc}
\usepackage[utf8x]{inputenc}
% \input{tikz.tex} 

\usepackage{amsmath} % nice math symbols
\usepackage{bm} % bold math
\usepackage{color} % change text color        

\usepackage{tikz}
\usetikzlibrary{decorations.pathmorphing} % for snake lines
\usetikzlibrary{matrix} % for block alignment
\usetikzlibrary{arrows} % for arrow heads
\usetikzlibrary{calc} % for manimulation of coordinates

% TikZ styles for drawing
\tikzstyle{block} = [draw,rectangle,thick,minimum height=3em,minimum width=7em,text width=7em,align=center]
\tikzstyle{sum} = [draw,circle,inner sep=0mm,minimum size=2mm]
\tikzstyle{connector} = [->,thick]
\tikzstyle{line} = [thick]
\tikzstyle{branch} = [circle,inner sep=0pt,minimum size=1mm,fill=black,draw=black]
\tikzstyle{guide} = []
\tikzstyle{snakeline} = [connector, decorate, decoration={pre length=0.2cm,
                         post length=0.2cm, snake, amplitude=.4mm,
                         segment length=2mm},thick, magenta, ->]

\renewcommand{\vec}[1]{\ensuremath{\boldsymbol{#1}}} % bold vectors
\def \myneq {\skew{-2}\not =} % \neq alone skews the dash
\usetikzlibrary{arrows,decorations.pathmorphing,backgrounds,fit,positioning,shapes.symbols,chains}
\begin{document}

  \begin{tikzpicture}[scale=1, auto, >=stealth']
\xdef\SIZE{7}
% \xdef\setka{1}
\xdef\opa{0.6}
\ifx\setka\undefined
    {}
    \else
      \if\setka1
        \draw[step=1.0,blue,thick, opacity=\opa] (0,0) grid (\SIZE,\SIZE);
        \draw[step=0.5,blue,very thin, opacity=\opa] (0,0) grid (\SIZE,\SIZE);

        \foreach \i in {0,1,...,\SIZE} {
            \draw (0,\i) node [left,  opacity=\opa] {\i};
            \draw (\i,0) node [below,  opacity=\opa] {\i};
        }
      \else
        {}
      \fi
    \fi 
	\node[block] (1) at (0,0) {Приемное устройство};
	\node [block, right=5em of 1] (2) {Линия связи};
	% \node [block, right=1cm of amp] (mod) {Модулятор};
	\node [block, right=5em of 2] (3) {Передающее устройство};
	\node [block, below=2em of 3] (6) {Приемное устройство};
	\node [block, below=2em of 6,fill=magenta!10] (9) {Получатель сообщений};
	\node [block, left=5em of 6] (5) {Линия связи};
	\node [block, left=5em of 5] (4) {Передающее устройство};

	\node [block, below=2em of 4,fill=magenta!10] (7) {Источник сообщений};
	\node [block, below=2em of 5] (8) {Источник помех};

    \draw[connector] (2) -- (1);
    \draw[connector] (3) -- (2);
    \draw[connector] (1) -- (4);
    \draw[connector] (6) -- (3);
    \draw[connector] (4) -- node[above]{$s(t)$} (5);
    \draw[connector] (5) -- node[above]{$u(t)$} (6);
    \draw[connector] (7) -- node[right]{$x(t)$} (4);
    \draw[connector] (8) -- node[right]{$n(t)$} (5);
	\draw[connector] (6) -- node[right]{$\tilde{x}(t)$} (9);

    \draw[dashed] (-1.75,-5em/2) rectangle node [above, yshift=1em]{Обратный канал} (10.75,1.5);
	% \draw[connector] (amp) -- (lpf);
	% \draw[connector] (lpf) -- (cyc)  node [pos=0.5,right,scale=1.1] {$\omega$};
	% \draw[connector] (cyc) -- (bpf) node [pos=0.5,right,scale=1.1] {$\omega,3\omega, \ldots$};
	% \draw[connector] (bpf) -- (adc1)  node [pos=0.5,right,fill=white,scale=1.1] {$3\omega$};
	% \draw[connector] (adc1) -- (comp);
	% \draw[connector] (adc2) -- (comp);

	% \draw[connector] ($(z.south)+(-1.5em,0)$) -- ($(adc2.north)+(-1.5em,0)$);

	% \draw[connector] ($(z.south)+(1.5em,0)$) |- ($(adc1.west)$);
	% \draw[connector] (T) -- (adc2)  node [pos=0.5,right,scale=1.1] {$T$};
	
	% \draw[connector] ($(cyc.west)+(0,1em)$) -- ($(z.east)+(0,1em)$) node [pos=0.5,above,scale=1.1] {$\omega$};
	% \draw[connector,<-] ($(cyc.west)+(0,-1em)$) -- ($(z.east)+(0,-1em)$) node [pos=0.5,below,scale=1.1] {$\omega,3\omega, \ldots$};

    % \small
    % % node placement with matrix library: 5x4 array
    % \matrix[ampersand replacement=\&, row sep=0.2cm, column sep=0.4cm] {
    %   %
    %   \node[block] (F1) {$\vec{u}_i = F_i(\{\widetilde{\vec{x}}_j\}_{j=1}^N)$}; \&
    %   \node[branch] (u1) {}; \&
    %   \&
    %   \node[block] (f1) {$\begin{matrix}
    %         \dot{\vec{x}}_i =
    %           f_i(\vec{x}_i,
    %               \textcolor{red}{\{\widetilde{\vec{x}}_j\}_{j \myneq i}},
    %               \vec{u}_i,
    %               t)\\
    %         \vec{y}_i =
    %           g_i(\vec{x}_i,
    %               \textcolor{blue}{\{\widetilde{\vec{x}}_j\}_{j \myneq i}},
    %               t)
    %       \end{matrix}$}; \& \\

    %   \&
    %   \&
    %   \&
    %   \node[block] (L1) {$\vec{e}_i(\vec{y}_i - \widetilde{\vec{y}}_i)$};\&
    %   \node [sum] (e1) {}; \\

    %   \&
    %   \&
    %   \node[sum] (v1) {}; \&
    %   \node[block] (o1) {$\begin{matrix}
    %         \dot{\widetilde{\vec{x}}}_i =
    %           \widetilde{f}_i(\widetilde{\vec{x}}_i,
    %                           \textcolor{red}{\{\widetilde{\vec{x}}_j\}_{j \myneq i}},
    %                           \vec{v}_i, t)\\
    %           \widetilde{\vec{y}}_i =
    %             g_i(\widetilde{\vec{x}}_i,
    %                 \textcolor{blue}{\{\widetilde{\vec{x}}_j\}_{j \myneq i}},
    %                 t)
    %       \end{matrix}$};
    %   \&
    %   \\
    %   \node[guide] (i1) {}; \& \& \& \& \\
    % };

    % % now link the nodes
    % \draw [line] (F1) -- (u1);
    % \draw [connector] (u1) -- node {$u_i$} (f1);
    % \draw [connector] (f1) -| node[near end] {$\vec{y}_i$} (e1);
    % \draw [connector] (e1) -- (L1);
    % \draw [connector] (L1) -| (v1);
    % \draw [connector] (v1) -- node {$\vec{v}_i$} (o1);
    % \draw [connector] (u1) |- (v1);
    % \draw [connector] (o1) -| node[pos=0.96] {$-$} node [near end, swap]
    %                   {$\widetilde{\vec{y}}_i$} (e1);
    % \draw [connector] (o1.south) -- ++(0,-.5cm) -| node [near start]
    %                   {$\widetilde{\vec{x}}_i$} ($(F1.south) + (0.4cm, 0em)$);
    
    % % draw the snake lines with offset (using the calc library)
    % \draw [snakeline] ($(i1) - (0.4cm, -1cm)$) -- node
    %   {$\{\widetilde{\vec{x}}_j\}_{j \myneq i}$} ($(F1.south) - (0.4cm, 0em)$);

    % \draw [snakeline, swap] ($(v1.east) - (1.0cm, 0.4cm)$) -- node
    %   {$\{\widetilde{\vec{x}}_j\}_{j \myneq i}$} ($(o1.west) - (0cm, 0.4cm)$);
    
    % \draw [snakeline, swap] ($(u1.east) + (0.1cm, -0.4cm)$) -- node
    %   {$\{\widetilde{\vec{x}}_j\}_{j \myneq i}$} ($(f1.west) - (0cm, 0.4cm)$);


  \end{tikzpicture}
\end{document}