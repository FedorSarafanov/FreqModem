\documentclass[a4paper,12pt]{article}

\usepackage{cmap}
\usepackage[T2A]{fontenc}
\usepackage[utf8x]{inputenc}
\usepackage[english, russian]{babel}

\usepackage{misccorr} % в заголовках появляется точка, но при ссылке на них ее нет
\usepackage{amssymb,amsfonts,amsmath,amsthm}  
\usepackage{indentfirst}
\usepackage[usenames,dvipsnames]{color} 
\usepackage[unicode,hidelinks]{hyperref}
% \hypersetup{%
%     pdfborder = {0 0 0}
% }
\usepackage{makecell,multirow} 
\usepackage{ulem}
\usepackage{graphicx,wrapfig}
\graphicspath{{img/}}
\usepackage{geometry}
\geometry{left=2cm,right=2cm,top=3cm,bottom=3cm,bindingoffset=0cm,headheight=15pt}
\usepackage{fancyhdr} 
% \linespread{1.2} 
\frenchspacing 
\renewcommand{\labelenumii}{\theenumii)} 
% \usepackage{caption}
%%%%%%%%%%%%%%%%%%%%%%%%%%%%%%%%%%%%%%%%%%%%%%%%%%%%%%%%%%%%%%%%%%%%%%%%%%%%%%%
%%%%%%%%%%%%%%%%%%%%%%%%%%%%%%%%%%%%%%%%%%%%%%%%%%%%%%%%%%%%%%%%%%%%%%%%%%%%%%%

\def\labauthor{Сарафанов Ф.Г., Платонова М.В.}
\def\labauthors{\labauthor}
\def\labnumber{1}
\def\labtheme{Частотный модем}

%%%%%%%%%%%%%%%%%%%%%%%%%%%%%%%%%%%%%%%%%%%%%%%%%%%%%%%%%%%%%%%%%%%%%%%%%%%%%%%
	%применим колонтитул к стилю страницы
\pagestyle{fancy} 
	%очистим "шапку" страницы
\fancyhead{} 
	%слева сверху на четных и справа на нечетных
\fancyhead[L]{\labauthors} 
	%справа сверху на четных и слева на нечетных
\fancyhead[R]{Отчёт по лабораторной работе №\labnumber} 
	%очистим "подвал" страницы
\fancyfoot{} 
	% номер страницы в нижнем колинтуле в центре
\fancyfoot[C]{\thepage} 
\renewcommand{\phi}{\varphi}
%%%%%%%%%%%%%%%%%%%%%%%%%%%%%%%%%%%%%%%%%%%%%%%%%%%%%%%%%%%%%%%%%%%%%%%%%%%%%%%

\usepackage{float}
\usepackage[mode=buildnew]{standalone}
\usepackage{tikz} 
% \usepackage{subcaption}
\usepackage{tikz,csvsimple}
\usetikzlibrary{scopes}
\usetikzlibrary{%
     decorations.pathreplacing,%
     decorations.pathmorphing,%
    patterns,%
    calc,%
    scopes,%
    arrows,%
    % arrows.spaced,%
}
\makeatletter
\newif\if@gather@prefix 
\preto\place@tag@gather{% 
  \if@gather@prefix\iftagsleft@ 
    \kern-\gdisplaywidth@ 
    \rlap{\gather@prefix}% 
    \kern\gdisplaywidth@ 
  \fi\fi 
} 
\appto\place@tag@gather{% 
  \if@gather@prefix\iftagsleft@\else 
    \kern-\displaywidth 
    \rlap{\gather@prefix}% 
    \kern\displaywidth 
  \fi\fi 
  \global\@gather@prefixfalse 
} 
\preto\place@tag{% 
  \if@gather@prefix\iftagsleft@ 
    \kern-\gdisplaywidth@ 
    \rlap{\gather@prefix}% 
    \kern\displaywidth@ 
  \fi\fi 
} 
\appto\place@tag{% 
  \if@gather@prefix\iftagsleft@\else 
    \kern-\displaywidth 
    \rlap{\gather@prefix}% 
    \kern\displaywidth 
  \fi\fi 
  \global\@gather@prefixfalse 
} 
\newcommand*{\beforetext}[1]{% 
  \ifmeasuring@\else
  \gdef\gather@prefix{#1}% 
  \global\@gather@prefixtrue 
  \fi
} 
\makeatother

\usepackage{booktabs}
\usepackage{pgfplots, pgfplotstable}

\usepackage[outline]{contour}
\usepackage{tocloft}
\renewcommand{\cftsecleader}{\cftdotfill{\cftdotsep}} % for parts
% \renewcommand{\cftchapleader}{\cftdotfill{\cftdotsep}} % for chapters
\usepackage{pgfplots,pgfplotstable,booktabs,colortbl}
\pgfplotsset{compat=newest}
\usepackage{physics}
\usepackage{mathtools}
\mathtoolsset{showonlyrefs=true}
\newcommand\Smat{\hat { \mathbf { S } }}

\begin{document}
\begin{titlepage}
\begin{center}

{\textsc{Нижегородский государственный университет имени Н.\,И. Лобачевского}}
\vskip 2pt \hrule \vskip 3pt
{\textsc{Радиофизический факультет}}

\vfill


{{\LARGE Отчет по лабораторной работе №\labnumber}\vskip 12pt {\Huge \bfseries \labtheme}}

	
\vspace{2cm}
{\large Работу выполнили студенты \\[-0.25em] 430 группы радиофизического факультата \\[0.5em] {\Large \bfseries \labauthor}}

% \vspace{0.5cm}
% {e-mail: sfg180@yandex.ru}

% \vspace{2cm}

\end{center}

\vfill
	
% \begin{flushright}
% 	{Выполнили студенты 430 группы\\ \labauthor}%\vskip 12pt Принял:\\ Менсов С.\,Н.}
% \end{flushright}
	
% \vfill
	
\begin{center}
	{Нижний Новгород, 6 марта -- \today}
\end{center}

\end{titlepage}
\tableofcontents
\newpage

\section*{Введение}
\addcontentsline{toc}{section}{Введение}

% \section{Радиотехнические системы передачи информации}

В этой работе мы исследуем \textbf{частотный модем}. Модем -- часть радиотехнической схемы передачи сообщений. Он представляет собой совокупность модулятора и демодулятора ЧМ-сигнала (см. рис. \ref{fig:2}). 

В радиотехнике для передачи сигнала по линии связи используются ВЧ-сигналы, хорошо по ней распространяющиеся. Такой сигнал сам по себе не несет информации, а для вложения сообщения в ВЧ-сигнал к сигналу применяют операцию модуляции, которая заключается в изменении одного или нескольких параметров переносчика по закону передаваемого сообщения. Устройство, осуществляющее эту операцию, называется \textbf{модулятором}.

В обратную сторону, при приеме сообщения, \textbf{демодулятором} осуществляется  демодуляция (детектирование) сигнала, которая заключается в преобразовании принятого модулированного сигнала, искаженного помехами, в модулирующий сигнал. 

\begin{figure}[H]
	\centering
	\includegraphics[scale=1]{plot/struct}
	\caption{Структурная схема системы передачи дискретных сообщений}
	\label{fig:2}
\end{figure} 

В частотном модеме используется сигнал с частотной модуляцией. Мы рассмотрим также фазовую модуляцию и спектры модулированных сигналов, а также способ получения частотной модуляции с помощью $RC$-генератора звукового диапазона за счет управления сопротивления канала сток--исток полевого транзистора модулирующим сигналом. 

В эксперименте мы изучим влияние изменения амплитуды или частоты модулирующего колебания на спектр частотно-модулированного (выходного сигнала модулятора) сигнала. 

\section{Краткая теория работы}
\subsection{Сигналы с угловой модуляцией}
В качестве колебания-переносчика используется гармоническое колебание высокой частоты. В процессе модуляции в несущем колебании $f(t)=U_{0} \cos (\omega t+\varphi)$ можно изменять амплитуду (амплитудная модуляция), а также либо частоту $\omega$ (частотная модуляция) либо фазу $\phi$ (фазовая модуляция), оставляя амплитуду постоянной. Поскольку аргумент $\psi=\omega t+\phi$ является полной фазой и определяет текущее значение фазового угла, такие сигналы получили название сигналов с угловой модуляцией.

\subsubsection{Фазовая модуляция (ФМ)}

Если полная фаза процесса $\Psi(t)=\omega_{0} t+k X(t)$, где $X(t)$ - сообщение, $k$ -
коэффициент	пропорциональности,	$\omega_0$ --
значение частоты в отсутствии сообщения $X(t)$, то имеем сигнал с фазовой модуляцией
\begin{equation}
	V_\text{фм}(t)=U_{0} \cos \left[\omega_{0} t+k X(t)\right]
\end{equation}
	
% \begin{figure}[H]

% \end{figure}

\begin{wrapfigure}[19]{r}{0.5\linewidth} 
	\vspace{-2ex}
	\centering
	\includegraphics[]{fig/fig2-1}
	\vspace{-1em}
	\caption{}
	\label{fig:2.1}
\end{wrapfigure}

Если сообщение $X(t)=const$, то ФМ-сигнал является простым высокочастотным сигналом.
Если $X(t)=A\cos\Omega t$, то с увеличением значений сообщения $X(t)$ полная фаза $y(t)$ растет во времени быстрее, чем по линейному закону. При уменьшении значений модулирующего сообщения происходит спад скорости роста $y(t)$ во времени (см. рис. \ref{fig:2.1}).



Предельное значение этого фазового сдвига называют девиацией фазы $\Delta \psi$. В общем случае, когда сообщение $X(t)$ изменяет знак, принято различать девиацию фазы вверх $\Delta \psi_v =kX_{max}$ и девиацию фазы вниз $\Delta \psi_n =kX_{min}$.

\subsubsection{Частотная модуляция (ЧМ)}
Мгновенная частота $\omega(t)$ сигнала с угловой модуляцией определяется как первая производная от полной фазы по времени, т.е. мгновенная частота - это скорость изменения полной фазы:
\begin{gather}
	\omega (t)=\dv{\psi}{t}
\end{gather}
Откуда, полная фаза равна:
\begin{equation}
	\label{eq2.1}
	%
	\psi(t)=\int\limits_{0}^{t} \omega(\tau) d \tau+\varphi_{0}
\end{equation}

где $\phi_0$ - начальная фаза в момент времени t=0.
При ЧМ - сигнале между сообщением $X(t)$ и мгновенной частотой $a(t)$ будет связь
вида
\begin{equation}
	\label{eq2.2}
	%
	\omega(t)=\omega_{0}+k X(t)
\end{equation}
Поэтому из \eqref{eq2.1} и \eqref{eq2.2}
\begin{equation}
	%\label{eq2.1}
	V_\text{чм}(t)=U_{0} \cos \psi(t)=U_{0} \cos \left[\int\limits_{0}^{t} \omega(\tau) d \tau+\varphi_{0}\right]
\end{equation}


\begin{equation}
	V_\text{чм}(t)=U_{0} \cos \left[\omega_{0} t+k \int\limits_{0}^{t} X(\tau) d \tau+\varphi_{0}\right]
\end{equation}


\begin{wrapfigure}[11]{l}{0.5\linewidth} 
	\vspace{-2ex}
	\centering
	\includegraphics[scale=1.1]{fig/fig2-2}
	\caption{}
	\label{fig:2.2}
\end{wrapfigure}
Если X(t) - достаточно гладкая функция, то внешне осциллограммы ФМ и ЧМ - сигналов не отличаются (рис.2.2). Однако имеет место принципиальная разница: фазовый сдвиг между ФМ - сигналом и с немодулированным пропорционален X(t), для ЧМ этот сдвиг пропорционален интегралу от $X(t)$. Т.е. ЧМ и ФМ - сигналы ведут себя по-разному при изменении частоты модуляции и амплитуды модулирующего колебания.

При ЧМ девиация частоты пропорциональна амплитуде НЧ - сигнала, в то же время девиация частоты $\Delta \omega$ не зависит от частоты $\Omega$ модулирующего сигнала.
\begin{figure}[H]
	\centering
	\includegraphics[]{fig/fig2-3}
	\caption{}
	\label{fig:2.3}
\end{figure}
При ФМ индекс модуляции $m=\frac{\Delta \omega}{\Omega} \sim A$ - амплитуде НЧ - сигнала независимо от
частоты модуляции. Как следствие этого, девиация частоты при фазовой модуляции линейно увеличивается с ростом частоты модулирующего сигнала (рис. \ref{fig:2.3}).

\subsubsection{Общие соображения о спектре сигналов с угловой модуляцией}
Если колебание $V(t)=U_{0} \cos [\omega t+\varphi(t)]$
получено с помощью ФМ, то $\phi(t)$ и X(t) полностью совпадают по форме и отличаются лишь постоянными коэффициентами. При этом , с точностью до постоянного коэффициента совпадают спектры функций $\phi(t)$ и X(t).
При ЧМ функция $\phi(t)$ является интегралом от передаваемого сообщения X(t). При ЧМ спектр функции $\phi(t)$ состоит из тех же компонент, что и спектр сообщения X(t), но с измененными амплитудами и фазами.

 Модулированное по углу колебание можно рассматривать как сумму двух квадратурных колебаний: косинусного uc и синусного us, каждое из которых модулировано только по амплитуде. 
\begin{equation}
\label{eq3.3}
	%
	V(t)=U_{0} \cos \varphi(t) \cos \omega_{0} t-U_{0} \sin \varphi(t) \sin \omega_{0} t=u_{c}(t)-u_{s}(t)
\end{equation}
 Закон AM для косинусного колебания определяется медленной функцией $\cos(t)$, для синусного - функцией $\sin(t)$. Но для определения спектра AM колебания достаточно сдвинуть на частоту оо спектр огибающей амплитуд. Следовательно, для нахождения спектра колебания $u(t)$, определяемого выражением \eqref{eq3.3}, необходимо найти сначала спектры функций $\cos(t)$ $\sin(t)$, т.е. спектры огибающих квадратурных колебаний.

Из приведенных рассуждений следует, что при одном и том же передаваемом сообщении спектр колебания, модулированного по углу, значительно сложнее, чем спектр модулированного по амплитуде. Действительно, т.к. $\cos(t)$ и $\sin(t)$ являются нелинейными функциями своего аргумента $\phi(t)$, то спектры этих колебаний могут существенно отличаться от спектра функции $\phi(t)$.


\subsubsection{Спектр ЧМ и ФМ при малых индексах модуляции} % (fold)
	
Задачу о представлении сигналов с угловой модуляцией посредством суммы гармонических колебаний несложно решить в том случае, когда $m\ll 1$. Для тонально-модулированного колебания $V(t)=U_{0} \cos \left[\omega_{0} t+m\sin \Omega t\right]$

\begin{equation}
	V(t)=U_{0} \cos (m \sin \Omega) \cos \omega_{0} t-U_{0} \sin (m \sin \Omega t) \sin \omega_{0} t_{0}
\end{equation}

при $m\ll 1$ $\cos (m \sin \Omega t) \approx 1$, $\sin (m \sin \Omega t) \approx m \sin \Omega t$

\begin{equation}
	V(t) \approx U_{0} \cos \omega_{0} t-U_{0} m \sin \Omega t \cdot \sin \omega_{0} t
\end{equation}

\begin{figure}[h!]
	% \vspace{-2ex}
	\centering
	\includegraphics[]{fig/fig2-4}
	\caption{}
	\label{fig:2.4}
	% \label{fig:figure1}
\end{figure}

\begin{equation}
	V(t) \approx U_{0} \cos \omega_{0} t+\frac{m U_{0}}{2} \cos \left(\omega_{0}+\Omega\right) t-\frac{m U_{0}}{2} \cos \left(\omega_{0}-\Omega\right) t
\end{equation}


Таким образом, при $m\ll 1$ в спектре сигнала с угловой модуляцией содержатся несущая и верхняя и нижняя боковые компоненты. Индекс m играет здесь такую же роль, как и в AM - сигнале.
Однако колебание нижних боковых частот имеет сдвиг по фазе 180°.

\subsubsection{Спектр сигнала с угловой модуляцией при произвольном значении индекса модуляции}
Итак, при тональной угловой модуляции
\begin{equation}
	U(t)=U_{0} \cos \left(\omega_{0} t+m \sin \Omega t\right)=U_{0} \operatorname{Re}\left(e^{j \omega_{0} t} \cdot e^{j m \sin \Omega t}\right)
\end{equation}
\begin{equation}
	e^{j m \sin Z}=\sum_{k=-\infty}^{\infty} J_{k}(m) e^{j k Z}
\end{equation}
где $m$ - любое вещественное число, $J_{k}(m)$ - функция Бесселя k порядка от аргумента $m$. 
Представляя экспоненту в виде ряда, получим модель ЧМ-ФМ сигнала  с любым значением индекса модуляции:
\begin{equation}
	U(t)=U_{0} \sum_{k=-\infty}^{\infty} J_{k}(m) \cos \left(\omega_{0}+k \Omega\right)
\end{equation}
При $m\ll 1$ ширина спектра ЧМ как и АМ равна $2\Omega$, при значении $m$ от 0.5 до 1 приобретает значение вторая пара боковых частот, поэтому ширина спектра равна $4\Omega$.
\begin{figure}[H]
	\centering
	\includegraphics[]{fig/fig2-5}
	\caption{}
	\label{fig:2.5}
	% \label{fig:figure1}
\end{figure}

\begin{figure}[H]
	\centering
	\includegraphics[]{fig/fig2-6}
	\caption{}
	\label{fig:2.6}
	% \label{fig:figure1}
\end{figure}
Фазы колебаний на рисунке \ref{fig:2.6} не учитываются, однако следует иметь в виду что, при нечетных $k$ амплитуды нижних боковых следует брать со знаком минус. 

Чем больше индекс $k$ функции Бесселя, тем протяженнее область аргументов, при которых эта функция мала. Важно отметить, что с ростом индекса модуляции $m$ расширяется полоса частот, занимаемая сигналом. Обычно полагают, что допустимо пренебречь всеми спектральными составляющими с номером $|k|>m+1$. Отсюда следует оценка практической ширины спектра с угловой модуляцией 
\begin{equation}
	\Pi_\text{пр}=2k\Omega=2(m+1)\Omega
	\label{eq:ocenka}
\end{equation}.

Как видно, реальные ЧМ и ФМ - сигналы характеризуются условием $m\gg1$, итак:
\begin{equation}
	\Pi_\text{пр} \approx 2 m \Omega=2 \Delta \omega
\end{equation}

\subsection{Частотная модуляция в автогенераторе}
Задачу получения ЧМ колебаний можно сформулировать как задачу создания генератора гармонических колебаний, частота которого должна изменяться в соответствии с законом изменения управляющего сигнала. Частота колебаний генератора
определяется резонансной частотой контура $\omega_{0}=\frac{1}{\sqrt{L C}}$ и, следовательно, для ее
изменения необходимо менять либо емкость C, либо индуктивность L.
		
Продифференцировав $\omega_0$, например, по C получим $\frac{d \omega_{0}}{d C}=-\frac{\omega_{0}}{2 C}$ или $\frac{\Delta \omega}{\omega_{0}} \approx-\frac{1}{2} \frac{\Delta C}{C_{0}}$.

Как видно, при малых изменениях частоты можно считать, что она пропорциональна емкости, т.е., желая получить модулированное колебание, следует изменить емкость (или индуктивность) контура в соответствии с передаваемым сообщением, поэтому контур должен содержать емкостной (или индуктивный) параметрический элемент.

Широко распространенным способом электронного управления является подключение к контуру варикапа, емкость которого зависит от напряжения, приложенного в направлении запирания перехода.

Упрощенная схема автогенератора с варикапом изображена на рис. \ref{fig:3.2}.
\begin{figure}[H]
	\centering
	\includegraphics[]{fig/fig3-1}
	\caption{}
	\label{fig:3.1}
	% \label{fig:figure1}
\end{figure}
\begin{figure}[H]
	\centering
	\includegraphics[]{fig/fig3-2}
	\caption{}
	\label{fig:3.2}
	% \label{fig:figure1}
\end{figure}
Разделительный конденсатор $C_p$, предназначен для развязки контура от источника $E_0$. Источник $E_0$ устанавливает начальную рабочую точку (НРТ) на вольт-фарадной характеристике варикапа. Блокировочный дроссель $L_\text{др}$ предназначен для того, чтобы ВЧ ток от автогенератора не проходил в источник ЭДС $e_{\Omega}(t)$.

На схеме замещения (рис. \ref{fig:3.2}б) $C_0$ - средняя емкость в отсутствии модулирующего колебания, $\Delta C(t)$ - вариация емкости в зависимости от  $e_{\Omega}(t)$. Сопротивление p-n перехода R, объемное сопротивление полупроводника г.

Если напряжение на емкости достаточно мало, то, как отмечено выше, нелинейный элемент можно трактовать как линейный параметрический. Принимая рабочий участок <<а-а>> зависимости $C(U)$ (рис. \ref{fig:3.1}) за прямую линию получим следующее.

Если управляющее напряжение меняется по закону $u_{y n p}=E_{0}+U_{c} \cos \Omega t$, то емкость меняется по закону $C=C_{0}\left(1+m_{c} \cos \Omega t\right)$.

\subsection{Демодуляция ЧМ-сигналов}

При частотной модуляции, как известно, полезное сообщение пропорционально отклонению мгновенной частоты сигнала от частоты несущего колебания $\omega(t)-\omega_0$, где $\omega(t)=\omega_{0}+\Delta \omega \cos \Omega t$:
\begin{equation}
	U_\text{чм}(t)=E_{0} \cos \left[\omega_{0}+\Delta \omega \cos \Omega t\right] t
\end{equation}

ЧМ детектирование можно осуществить преобразованием ЧМ-сигнала в неглубокий АМ-сигнал и дальнейшим амплитудным детектированием. 
Для этого нужно подать ЧМ-сигнал на линейный частотный фильтр, настроенный таким образом, чтобы несущая частота ЧМ-сигнала попадала на линейный участок фильтра. АЧХ полосового фильтра можно разложить в ряд:
\begin{equation}
	\label{eq4.1}
	|K(j \omega)|=
	\qty|K(j \omega_{0})|+
	\qty|\dv{K(j\omega_0)}{\omega}|\cdot\qty(\omega(t)-\omega_0)+\ldots
\end{equation}

Тогда на выходе фильтра получится сигнал со сложной амплитудно-угловой модуляцией с мгновенной амплитудой переменной составляющей
\begin{equation}
	V_\text{вых}(t)=b_{0}\cdot\qty|\dv{K(j\omega_0)}{\omega}|\cdot \Delta \omega \cos \Omega t,
\end{equation}
где $b_0$ -- постоянный коэффициент. Окончательная обработка проводится обычным AM детектором, включаемым на выходе фильтра.

Этот метод имеет недостаток, обусловленный малым диапазоном линейности характеристики детектирования и необходимостью настройки на частоту, отличную от частоты немодулированного колебания ($\omega_{\text{рез}} \neq \omega_{0}$).

Для устранения этого недостатка совмещают в одной схеме два контура и два амплитудных детектора таким образом, что контуры настроены на частоты, симметрично смещенные относительно несущей, и при этом выходы двух детекторов соединены так, что их выходные напряжения вычитаются друг из друга. В этом случае детекторная характеристика будет иметь достаточно протяженный, почти линейный участок.

\subsection{$RC$--генератор низкочастотных гармонических колебаний}

В нашей установке был использован $RC$-автогенератор гармонических колебаний звукового диапазона в составе частотного модулятора.

Рассмотрим уравнение автогенератора с $LC$-контуром:
\begin{gather}
	\label{eq5.1}
	\dv[2]{U}{t}+
	\qty(
		\frac{R}{L} - 
		\frac{M}{LC}\dv{\varphi(U)}{U}
	)\dv{U}{t}+
	\omega_r^2U=0
\end{gather}
Это уравнение имеет второй порядок. Действием обратной связи, реализованной на трансформаторной связи с взаимной индуктивностью $М$, коэффициент при первой производной 
$\qty(
	\frac{R}{L} - 
	\frac{M}{LC}\dv{\varphi(U)}{U}
)$  
обращается в нуль в стационарном режиме и отрицательна в режиме самовозбуждения. Для создания $RC$-автогенератора нужно так составить схему, чтобы она описывалась таким же дифференциальным уравнением второго порядка, как и \eqref{eq5.1}. Рассмотрим схему, представленную на рис. \ref{fig:5.1}. 
\begin{figure}[H]
	\centering
	\includegraphics[]{fig/fig5-1.pdf}
	\caption{}
	\label{fig:5.1}
\end{figure}
Дифференциальное уравнение этой схемы
\begin{gather}
	\label{eq5.2}
	%
	\dv{U_2}{t}+
		2 \alpha U_2 + 
		\omega_0^2\int\limits_0^t U_2\, \dd t=
		\frac{U_1}{\tau_{12}}
	\quad \Rightarrow \quad
	%
	\dv[2]{U_2}{t}+
		2 \alpha \dv{U_2}{t} + 
		\omega_0^2 U_2 =
		\frac{1}{\tau_{12}}	\dv{U_1}{t},\\
	%
	\text{где} \quad
	\omega_0^2=\frac{1}{R_1 R_2 C_1 C_2}, \quad
	2 \alpha=\frac{1}{\tau_1}+\frac{1}{\tau_2}+\frac{1}{\tau_{12}}, \quad
	\tau_1=R_1C_1, \quad \tau_2=R_2C_2, \quad \tau_{12}=R_1C_2.
\end{gather}
Левая часть уравнения \eqref{eq5.2} совершенно аналогична левой части уравнения \eqref{eq5.1}. Если теперь в эту схему внести нелинейный усилитель и охватить всю схему обратной связью, как показано на рис.  \ref{fig:5.2}, то полученная схема сможет генерировать гармонические колебания на одной из частот, определяемых уравнением \eqref{eq5.2}.
\begin{figure}[H]
	\centering
	\includegraphics[]{fig/fig5-2.pdf}
	\caption{}
	\label{fig:5.2}
\end{figure}
Найдем дифференциальное уравнение схемы рис.  \ref{fig:5.2}, положив в ней $K(\omega)=\frac{U_1}{U_2}$, где $U_1$ -выходное напряжение усилителя $К(\omega)$, a $U_2$ -- входное. 
Из этого уравнения можно найти 
\begin{equation}
	U_1=K(\omega)\cdot U_2, \qquad
	\dv{U_1}{t}=K\dv{U_2}{t}.
\end{equation}
Подставляя полученное соотношение в \eqref{eq5.2}, получим
\begin{gather}
	\label{eq5.3}
	%
	\dv[2]{U_2}{t}+
		2 \alpha \dv{U_2}{t} -
		\frac{K}{\tau_{21}}	\dv{U_2}{t}+ 
		\omega_0^2 U_2 =
		0, 
	%
	\\ \text{или} \quad
	%
	\dv[2]{U_2}{t}+
		\qty(
			\frac{1}{\tau_1}+
			\frac{1}{\tau_2}+
			\frac{1-K}{\tau_{21}}
		)\dv{U_2}{t}+ 
		\omega_0^2 U_2=0, 
	\quad \text{где} \quad
	\tau_{21}=R_2C_1.
\end{gather}

Уравнение \eqref{eq5.3} совершенно аналогично уравнению \eqref{eq5.1} с $LC$-контуром. Для осуществления самовозбуждения необходимо сделать отрицательным коэффициент при первой производной
\begin{equation}
	\left(\frac{1}{R_{1} C_{1}}+\frac{1}{R_{2} C_{2}}+\frac{1-K}{R_{2} C_{1}}\right)<0
\end{equation}

Типичная схема RC-автогенератора приведена на рис.  \ref{fig:5.3}(а), а его эквивалентная схема с разомкнутой цепью обратной связи - на рис. \ref{fig:5.3}(б). Здесь $K_0$ - идеальный усилитель с вещественным и положительным коэффициентом усиления $K_0$. Выход усилителя соединяется с его входом через пассивный четырехполюсник, выделенный на рисунке пунктирной рамкой и представляющий собой цепь положительной обратной связи. Передаточная функция этой цепи равна
\begin{equation}
	\beta(j \omega)=
	\cfrac{
		\cfrac{
			R_{2}/{j \omega C_{2}}
		}{
			R_{2}+{1}/{j \omega C_{2}}}
	}{
		% \qty(
			R_1+\cfrac{1}{j \omega C_1}+
		% )+
		\cfrac{
			R_2/{j \omega C_2}
		}{
			R_2+1/{j \omega C_2}
		}
	}
\end{equation}
или в переменных Лапласа
\begin{gather}
 	\label{eq5.5}
	 \beta(p)=\frac{p \tau_{21}}{\left(1+p \tau_{1}\right)\left(1+p \tau_{2}\right)+p \tau_{21}}.
\end{gather}

Характеристическое уравнение автогенератора в общем виде имеет вид
\begin{gather}
	\label{eq5.6}
	K(j \omega) \beta(j \omega)-1=0 \qq{или} K(p) \beta(p)=1
\end{gather}
Для схемы рис. \ref{fig:5.3} с передаточной функцией цепи положительной обратной связи $\beta(p)$ в соответствии с \eqref{eq5.5} характеристическое уравнение имеет вид
\begin{gather}
	\label{eq5.7}
	a_{2} p^{2}+a_{1} p+1=0, \qq {где} a_{2}=\tau_{1} \tau_{2}, \quad a_{1}=\tau_{1}+\tau_{2}-\tau_{21}\left(K_{0}-1\right)
\end{gather}
\begin{figure}[H]
	\centering
	\includegraphics[]{fig/fig5-3.pdf}
	\caption{$RC$-генератор и его эквивалентная схема}
	\label{fig:5.3}
\end{figure}
Условия самовозбуждения автогенератора состоит в том, чтобы коэффициент $a_1$ был меньше нуля. Отсюда находится условие, налагаемое на коэффициент усиления $K_0$:
\begin{gather}
	\label{eq5.8}
	K_{0}>1+\frac{R_{1} C_{1}+R_{2} C_{2}}{R_{2} C_{1}}
\end{gather}
Частота генерации определяется мнимой частью корней характеристического уравнения \eqref{eq5.7}. Для стационарного режима автогенератора коэффициент $a_1$ этого уравнения равен нулю, и уравнение принимает вид
\begin{gather}
	\label{eq5.9}
	p^{2} \tau_{1} \tau_{2}+1=0.
\end{gather}
Мнимые части корней этого уравнения и частота генерации равны
\begin{gather}
	\label{eq5.10}
	p_{1,2}=\pm j \sqrt{\frac{1}{\tau_{1}\tau_{2}}}, \qquad
	\omega_{0}=\frac{1}{\sqrt{R_{1} R_{2} C_{1} C_{2}}}
\end{gather}
Обычно выбирают $R_{1}=R_{2}=R$ и $C_{1}=C_{2}=C$. При этом передаточная характеристика \eqref{eq5.5} принимает вид
\begin{gather}
	\label{eq5.11}
	\beta(j \omega)=\cfrac{
		j\qty[\cfrac{\omega}{\omega_0}]
	}{
		1+3j\qty[\cfrac{\omega}{\omega_0}]-
		\qty[\cfrac{\omega}{\omega_0}]^2
	}\,\,, \qq{где} 
	\omega_0=\frac{1}{RC}.
\end{gather}
Соответствующие амплитудно-частотная и фазо-частотная характеристики представлены на рис. \ref{fig:5.4} (а) и (б) соответственно.
\begin{figure}[H]
	\centering
	\includegraphics[]{fig/fig5-4.pdf}
	\caption{АЧХ и ФЧХ генератора}
	\label{fig:5.4}
\end{figure}
Условие возбуждения такого генератора на частоте $\omega_0$ при этом переходит в $K_0 > 3$. 

\section{Исследование частотного модулятора}
Мы займемся исследованием принципа действия частотного модулятора, получением характеристик частотного модулятора при воздействии на его вход гармонического сигнала (тональная модуляция) и исследование формы и спектра сигналов с частотной модуляцией.

\subsection{Схема работы и измерительная аппаратура}
В данной работе используется универсальный лабораторный стенд со сменным блоком <<частотный модем>>, упрощённая принципиальная схема которого приведена на рис. \ref{fig:6.1}. В первой части работы объектом исследования является левая часть схемы (между гнёздами КТ 1 и КТ 2).
\begin{figure}[H]
	\centering
	\includegraphics[scale=1]{fig/fig6-1}
	\caption{Упрощенная схема частотного модема}
	\label{fig:6.1}
\end{figure}

Частотный модулятор представляет собой $RC$-генератор, состоящий из двухкаскадного резистивного усилителя (А1) и фазобалансной цепи (ФБЦ), обеспечивающей положительную обратную связь. Частота генерации зависит от параметров ФБЦ - С3, С4 и сопротивлений каналов сток-исток полевых транзисторов VT1 и VT2. Сопротивление канала ($R_\text{си}$) зависит от управляющего напряжения, приложенного к затвору. Таким образом, полевой транзистор в ФБЦ является параметрическим элементом, управляемым модулирующим напряжением. Напряжение смещения ($E_\text{см}$), являющееся постоянной составляющей модулирующего сигнала, позволяет установить несущую частоту модулированного сигнала, а переменная составляющая, т.е. сам модулирующий сигнал, поданный на гнезда КТ 1, обеспечивает девиацию частоты $\Delta f_\text{max}$, зависящую от амплитуды модулирующего сигнала. Выходом частотного модулятора являются гнезда КТ 2.

В схеме модулятора имеется блок автоматической регулировки усиления, поддерживающий постоянную амплитуду ЧМ-сигнала (на схеме не показан).

В качестве источника модулирующего сигнала используется встроенный диапазонный генератор НЧ с цифровой индикацией частоты выходного гармонического сигнала, подключаемый к входу модулятора. Для контроля амплитуды модулирующего сигнала используется встроенный вольтметр. Измерение частот, анализ осциллограмм и спектра сигналов производится на двухлучевом цифровом осциллографе TDS 2002.


\subsection{Измерение статической модуляционной характеристики}

Статическая модуляционная характеристика (СМХ) представляет собой зависимость частоты модулированного сигнала при отсутствии модулирующего сигнала от рабочей точки
\begin{equation}
	f=f(E_\text{см})
\end{equation}

Последовательно задавая $E_\text{см}$ значения $0,-0.5,\ldots,-6.5$ вольт, с помощью осциллографа, подключенного к выходу модулятора, определили соответствующие значения частоты, свели в таблицу и построили график (см. рис. \ref{fig:exp1}).

\begin{figure}[H]
\centering
\begin{minipage}{0.99\linewidth} 
\begin{minipage}{0.3\linewidth} 
\centering
\begin{table}[H]
	\label{tab:1}
	\centering
	\vspace{5pt}
	\includegraphics[scale=1.2]{plot/tab1.pdf}
	\vspace{4.5em}
\end{table}
\end{minipage} 
\begin{minipage}{0.7\linewidth} 
	\centering
	\includegraphics[width=0.9\textwidth]{plot/ef.pdf}
	% \caption{Начало измерений}
\end{minipage}
\end{minipage}
\vspace{-2em}
\label{fig:exp1}
\caption{Зависимость частоты от рабочей точки}
\end{figure}

Из графика и таблицы мы определили положение рабочей точки $E_\text{см}=-3.5$ В, несущую частоту $f_0=17.62$ кГц и тангенс угла наклона линейного участка СМХ, равный коэффициенту $K_\text{чм}$ модулятора -- $K_\text{чм}=1220$, и границы линейного участка $f_\text{min}\ldots f_\text{max}=14.05\ldots22.02$ кГц
				
\subsection{Спектр ЧМ при варьировании амплитуды модулирующего сигнала}

% \paragraph{Расчет амплитуд модулирующего сигнала.} 
По ряду заданных значений $M_\text{чм}=0,0.1,0.5,1,2.4,3.8$ мы рассчитали амплитуды модулирующих сигналов, а затем и действующие значения $U_c=0.707\cdot U_{mc}$, выбрав $F_\text{мод}$ = 500 Гц:

\begin{equation}
	\Delta f_\text{max} = M_\text{чм}\cdot F_\text{мод}
\end{equation}

\begin{equation}
	U_{mc}=\frac{\Delta f_\text{max}}{K_\text{чм}}=
	F_\text{мод}\cdot\frac{M_\text{чм}}{K_\text{чм}}=500\cdot\frac{M_\text{чм}}{1220}
\end{equation}
\vspace{1em}					
Мы подключили встроенный НЧ генератор (выход 0 дБ) к входу модулятора (гнездо КТ1), соединили общие точки (землю) встроенного НЧ генератора и модулятора. К входу модулятора также подключили вольтметр переменного напряжения (гнездо <<+>>).

Общую точку модулятора соединили с отрицательным входом вольтметра. 
Установили частоту генератора $F_\text{мод} = 500$ Гц (диапазон 2).

Последовательно устанавливая значения $U_c$ по показаниям вольтметра по значениям $U_mc$, рассчитанным ранее, регулятором амплитуды выхода НЧ генератора (ниже ручки точной регулировки частоты), получили на осциллографе, подключённом к выходу модулятора (гнездо КТ 2) спектры ЧМ-сигналов в режиме БПФ. 

Установив развертку 500 Гц/дел, мы измерили практическую ширину спектра  $2\Delta f^*$, учитывая только компоненты спектра с  амплитудами не менее $-30$ дБ от максимальных. Полученные спектрограммы показаны на рис. \ref{fig:M}, а данные сведены в таблицу \ref{tab:M}.

% \begin{figure}[h!]
% 	\centering
% 	\includegraphics[width=0.5\textwidth]{photo/start.jpg}
% 	\caption{Начало измерений}
% 	\label{fig:start}
% \end{figure}

\begin{table}[H]
	\caption{Влияние амплитуды модулирующего сигнала на спектр ЧМ}
	\vspace{1em}
	\label{tab:M}
	\centering
	\includegraphics[scale=1.2]{plot/tab3.pdf}
\end{table}

\paragraph{Теоретический расчет спектра при заданной $F_\text{мод}$.} Для теоретического расчета воспользуемся приближенной формулой \eqref{eq:ocenka}:
\begin{equation}
	\Pi_\text{пр}=2(m+1)\Omega
\end{equation}
В частности, при $F_{\text{мод}}=500\text{ Гц}$, $U_c=0.695$, $M_{\text{чм}}=2.4$  получаем
\begin{equation}
	\Pi_\text{пр}=2\cdot3.4\cdot500 \text{ Гц}=3.4 \text{ кГц}
\end{equation}
При этом экспериментально полученное значение ширины спектра составило 3 кГц.



\begin{figure}[H]
	\centering
	\includegraphics[width=\textwidth]{plot/M}
	\vspace{-2em}
	\caption{Влияние амплитуды модулирующего сигнала на спектр ЧМ}
	\label{fig:M}
\end{figure}
\paragraph{Вывод.} % (fold)
	Иходя из данных эксперимента, можно сделать вывод, что при увеличении амплитуды модулирующего сигнала спектр на выходе модулятора расширяется, т. е. влияние боковых частот возрастает. Причем амплитуда несущего колебания уменьшается. Такой результат согласуется с теорией, согласно которой полоса частот, занимаемая выходным сигналом, расширяется с ростом индекса модуляции.
% paragraph вывод (end)

\subsection{Спектр ЧМ при варьировании частоты модуляции}

Схема подключения установки осталась той же, что и в предыдущем эксперименте. Мы установили значение $U_c$, соответствующее индексу модуляции $M_\text{чм}=2.4$ и больше его не изменяли.

Последовательно устанавливая частоты модуляции (табл. \ref{tab:tablename}), мы получили спектрограммы соответствующих ЧМ-сигналов. В таблицу внесли значения $2\Delta f^*$.  При выполнении этого задания практическую ширину спектра определяли по спектральным компонентам, которые составляли не менее $-20$ дБ от максимальных амплитуд.


\begin{figure}[H]
	\centering
	\includegraphics[width=\textwidth]{plot/F}
	\caption{Влияние частоты модулирующего сигнала на спектр ЧМ}
	\label{fig:figureF}
\end{figure}

\begin{table}[H]
	\caption{Влияние частоты модулирующего сигнала на спектр ЧМ}
	\vspace{1em}
	\label{tab:tablename}
	\centering
	\includegraphics[scale=1.2]{plot/tab4.pdf}
\end{table}

Для каждого измерения мы рассчитали $M_\text{чм}$ по формуле
\begin{equation}
	M_\text{чм}=\frac{\Delta f_\text{max}}{F_\text{мод}}
\end{equation}

\paragraph{Теоретический расчет спектра при заданной $F_\text{мод}$.} Для теоретического расчета воспользуемся приближенной формулой \eqref{eq:ocenka}:
\begin{equation}
	\Pi_\text{пр}=2(m+1)\Omega
\end{equation}
В частности, при $F_{\text{мод}}=250\text{ Гц}$, $U_c=0.695$, $M_{\text{чм}}=2.4$  получаем
\begin{equation}
	\Pi_\text{пр}=2\cdot3.4\cdot250 \text{ Гц}=1.7 \text{ кГц}
\end{equation}
При этом экспериментально полученное значение ширины спектра составило 2 кГц.
\paragraph{Вывод.}
При увеличении частоты модулирующего сигнала (при постоянной амплитуде) количество спектральных компонент модулируемого колебания на выходе уменьшается, амплитуда несущего колебания при этом возрастает. Этот факт согласуется с теорией, если учесть, что значение индекса модуляции при этом уменьшается.

\subsection{Форма модулированного и немодулированного сигналов}

Мы соединили второй вход двухлучевого осциллографа с входом модулятора (для этого отключили вольтметр, сохраняя соединение с генератором). На первый вход осциллографа подали выходной сигнал модулятора.

\begin{figure}[H]
	\centering
	\includegraphics[width=\textwidth]{plot/mod}
	\vspace{-2em}
	\caption{Частота модулированного сигнала}
	\label{fig:figure4}
\end{figure}

Установили максимальную частоту модуляции $F_\text{мод}$ для второго поддиапазона встроенного НЧ генератора входного сигнала (2.2 кГц), а уровень сигнала увеличивали до тех пор, пока на осциллограмме выходного сигнала не появилась паразитная амплитудная модуляция, после чего несколько уменьшили входной сигнал так, чтобы огибающая ЧМ-сигнала снова стала почти ровной.

Мы зафиксировали осциллограммы (рис. \ref{fig:figure4}) на входе и выходе частотного модулятора, нажав сначала кнопку <<auto set>>, а затем кнопку <<run/stop>> на передней панели осциллографа. 

Оценили частоты выходного модулированного колебания на интервалах времени, соответствующих максимальному уровню модулирующего колебания -- 18.52 кГц и минимальному -- 16.13 кГц.


\section*{Заключение}
\addcontentsline{toc}{section}{Заключение}
Проведено исследование принципа действия частотного модулятора. В частности:
\begin{enumerate}
	\item Снята СМХ модулятора и определен его оптимальный режим работы: $E_{\text{см}}=-3,5 {\text{ В}}$, $f_0=17,62 {\text{кГц}}$
	\item Исследована зависимость спектра ЧМ-сигнала на выходе модулятора от амплитуды модулирующего колебания при неизменной частоте модулирующего сигнала (в нашем случае 500 Гц). Полученная в эксперименте зависимость согласуется с теорией. Так, для $F_{\text{мод}}=500\text{ Гц}$, $U_c=0.695$, $M_{\text{чм}}=2.4$ 
\begin{equation}
	\Pi_\text{теор}=3.4 \text{ кГц}, \quad 
	\Pi_\text{эксп}=3 \text{ кГц}
\end{equation}
	\item  Исследована зависимость спектра ЧМ-сигнала на выходе модулятора от частоты модулирующего колебания, при постоянной амплитуде модулирующего колебания. Экспериментальные данные хорошо согласуются с теоретическим расчетом. Так, для $F_{\text{мод}}=250\text{ Гц}$, $U_c=0.695$, $M_{\text{чм}}=2.4$ 
\begin{equation} 
\Pi_\text{эксп}=1.7 \text{ кГц}, \quad \Pi_\text{теор}=2 \text{ кГц} 
\end{equation}
	\item С помощью моментального снимка — совмещенных осциллограмм ЧМ-сигнала и модулирующего сигнала проверили изменение частоты в ЧМ-сигнале: на минимуме модулирующего сигнала и на максимуме соответственно частота ЧМ-сигнала составила 
\begin{equation} 
f_\text{мин мод}=18.52\text{ кГц}, \quad 
f_\text{макс мод}=16.13\text{ кГц} 
\end{equation}
	
\end{enumerate}


\addcontentsline{toc}{section}{Список литературы}
\begin{thebibliography}{99}
\bibitem{a} Орлов И. Я. Курс лекций по основам радиоэлектроники. Н.Новгород: ННГУ, 2005.
\bibitem{b} Канаков В.А. Методические указания к лабораторной работе <<Частотный модем>>. Н.Новгород: ННГУ, 2008.
\end{thebibliography}


\end{document}
