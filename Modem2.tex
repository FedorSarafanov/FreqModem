\documentclass[a4paper,12pt]{article}

\usepackage{cmap}
\usepackage[T2A]{fontenc}
\usepackage[utf8x]{inputenc}
\usepackage[english, russian]{babel}

\usepackage{misccorr} % в заголовках появляется точка, но при ссылке на них ее нет
\usepackage{amssymb,amsfonts,amsmath,amsthm}  
\usepackage{indentfirst}
\usepackage[usenames,dvipsnames]{color} 
\usepackage[unicode,hidelinks]{hyperref}
% \hypersetup{%
%     pdfborder = {0 0 0}
% }
\usepackage{makecell,multirow} 
\usepackage{ulem}
\usepackage{graphicx,wrapfig}
\graphicspath{{img/}}
\usepackage{geometry}
\geometry{left=2cm,right=2cm,top=3cm,bottom=3cm,bindingoffset=0cm,headheight=15pt}
\usepackage{fancyhdr} 
\linespread{1.2} 
\frenchspacing 
\renewcommand{\labelenumii}{\theenumii)} 
% \usepackage{caption}
%%%%%%%%%%%%%%%%%%%%%%%%%%%%%%%%%%%%%%%%%%%%%%%%%%%%%%%%%%%%%%%%%%%%%%%%%%%%%%%
%%%%%%%%%%%%%%%%%%%%%%%%%%%%%%%%%%%%%%%%%%%%%%%%%%%%%%%%%%%%%%%%%%%%%%%%%%%%%%%

\def\labauthor{Сарафанов Ф.Г., Платонова М.В.}
\def\labauthors{\labauthor}
\def\labnumber{1}
\def\labtheme{Частотный модем}

%%%%%%%%%%%%%%%%%%%%%%%%%%%%%%%%%%%%%%%%%%%%%%%%%%%%%%%%%%%%%%%%%%%%%%%%%%%%%%%
	%применим колонтитул к стилю страницы
\pagestyle{fancy} 
	%очистим "шапку" страницы
\fancyhead{} 
	%слева сверху на четных и справа на нечетных
\fancyhead[L]{\labauthors} 
	%справа сверху на четных и слева на нечетных
\fancyhead[R]{Отчёт по лабораторной работе №\labnumber} 
	%очистим "подвал" страницы
\fancyfoot{} 
	% номер страницы в нижнем колинтуле в центре
\fancyfoot[C]{\thepage} 

%%%%%%%%%%%%%%%%%%%%%%%%%%%%%%%%%%%%%%%%%%%%%%%%%%%%%%%%%%%%%%%%%%%%%%%%%%%%%%%

\usepackage{float}
\usepackage[mode=buildnew]{standalone}
\usepackage{tikz} 
% \usepackage{subcaption}
\usepackage{tikz,csvsimple}
\usetikzlibrary{scopes}
\usetikzlibrary{%
     decorations.pathreplacing,%
     decorations.pathmorphing,%
    patterns,%
    calc,%
    scopes,%
    arrows,%
    % arrows.spaced,%
}
\makeatletter
\newif\if@gather@prefix 
\preto\place@tag@gather{% 
  \if@gather@prefix\iftagsleft@ 
    \kern-\gdisplaywidth@ 
    \rlap{\gather@prefix}% 
    \kern\gdisplaywidth@ 
  \fi\fi 
} 
\appto\place@tag@gather{% 
  \if@gather@prefix\iftagsleft@\else 
    \kern-\displaywidth 
    \rlap{\gather@prefix}% 
    \kern\displaywidth 
  \fi\fi 
  \global\@gather@prefixfalse 
} 
\preto\place@tag{% 
  \if@gather@prefix\iftagsleft@ 
    \kern-\gdisplaywidth@ 
    \rlap{\gather@prefix}% 
    \kern\displaywidth@ 
  \fi\fi 
} 
\appto\place@tag{% 
  \if@gather@prefix\iftagsleft@\else 
    \kern-\displaywidth 
    \rlap{\gather@prefix}% 
    \kern\displaywidth 
  \fi\fi 
  \global\@gather@prefixfalse 
} 
\newcommand*{\beforetext}[1]{% 
  \ifmeasuring@\else
  \gdef\gather@prefix{#1}% 
  \global\@gather@prefixtrue 
  \fi
} 
\makeatother

\usepackage{booktabs}
\usepackage{pgfplots, pgfplotstable}

\usepackage[outline]{contour}
\usepackage{tocloft}
\renewcommand{\cftsecleader}{\cftdotfill{\cftdotsep}} % for parts
% \renewcommand{\cftchapleader}{\cftdotfill{\cftdotsep}} % for chapters
\usepackage{pgfplots,pgfplotstable,booktabs,colortbl}

% \renewcommand{\arraystretch}{1.5} 

\pgfkeys{/pgf/number format/.cd,
		fixed,  1000 sep={\,}}

\pgfplotstableset{
	% multicolumn names, % allows to have multicolumn names
	% header=has colnames,
	% dec sep align,
	col sep=tab, % the seperator in our .csv file
	% fixed zerofill, 
	% precision=4,
	columns/1/.style={
		column name={1},
		string type,
	},	
	columns/2/.style={
		column name={2},
		string type,
	},	
	columns/3/.style={
		column name={3},
		string type,
	},	
	columns/Skm/.style={
		column name={$S{km}$},
		string type,
	},	
	columns/S/.style={
		column name={$S{km}$},
		string type,
		column type = {l},
	},	
	columns/N/.style={
		column name={№},
		precision=0,
		% fixed zerofill, 	
		% % column type/.add={|}{},
	},
	columns/Imax/.style={
		column name={$I{\max}$},
		dec sep align,
		precision=0,
	},
	columns/Imin/.style={
		column name={$I{\min}$},
		dec sep align,
		precision=1,
	},
	columns/Gamma/.style={
		column name={$|\Gamma|$},
		dec sep align,
		precision=2,
	},
	columns/phi/.style={
		column name={$\varphiH$},
		dec sep align,
		precision=2,
	},
	columns/zmin/.style={
		column name={$z{\min}$, см},
		dec sep align,
		precision=2,
	},
	empty cells with={\textbf{--}},
	every head row/.style={
	before row={\toprule},
	after row={
		\midrule}
		},
	every last row/.style={after row=\bottomrule},
	every row/.style={after row=\midrule}, 
	create on use/Gamma/.style={
	    create col/expr={
	    	(sqrt(\thisrow{Imax}/\thisrow{Imin})-1)/(sqrt(\thisrow{Imax}/\thisrow{Imin})+1)%*\thisrow{ul5}/\thisrow{u}
	    }
	},
	create on use/phi/.style={
	    create col/expr={
	    	4*pi*abs(\thisrow{zmin}-5.145)/5.47-pi
	    }
	},
	columns={N,Skm,1,2,3,Imax,Imin,zmin,Gamma,phi,S},		
	% dec zerofill
	% fixed,fixed zerofill,
	% precision=3
	% every even column/.style={%
	% 	% column type/.add={>{\columncolor[gray]{.8}}}{}
	% },
	% every even row/.style={%
	% 	before row={\rowcolor[gray]{0.95}},
	% },	
	% every head row/.style={
 %        before row={
 %        	& & \multicolumn{3}{c}{Номер входа} & \\ \toprule
 %        },
 %        after row=\midrule
	% },
	}%

\pgfplotsset{compat=newest}
\usepackage{physics}
\usepackage{mathtools}
\mathtoolsset{showonlyrefs=true}
\newcommand\Smat{\hat { \mathbf { S } }}

% \begin{figure}[tb]
% 	\centering
% 	\includegraphics[]{example-image-a}
% 	% \label{fig:figure1}
% 	\caption{}
% \end{figure}
\begin{document}
\begin{titlepage}
\begin{center}

{\textsc{Нижегородский государственный университет имени Н.\,И. Лобачевского}}
\vskip 2pt \hrule \vskip 3pt
{\textsc{Радиофизический факультет}}

\vfill


{{\LARGE Отчет по лабораторной работе №\labnumber}\vskip 12pt {\Huge \bfseries \labtheme}}

	
\vspace{2cm}
{\large Работу выполнили студенты \\[-0.25em] 430 группы радиофизического факультата \\[0.5em] {\Large \bfseries \labauthor}}

% \vspace{0.5cm}
% {e-mail: sfg180@yandex.ru}

% \vspace{2cm}

\end{center}

\vfill
	
% \begin{flushright}
% 	{Выполнили студенты 430 группы\\ \labauthor}%\vskip 12pt Принял:\\ Менсов С.\,Н.}
% \end{flushright}
	
% \vfill
	
\begin{center}
	{Нижний Новгород, 6 марта -- \today}
\end{center}

\end{titlepage}
\tableofcontents
\newpage
\section{Радиотехнические системы передачи информации}

Под системой передачи информации (рис.	1.1) понимают совокупность
технических средств, предназначенных для передачи информации, включая источник сообщений и получателя сообщений.

Источник сообщений - это устройство, осуществляющее выбор сообщений из ансамбля сообщений. Им может быть датчик, ЭВМ и т.п. В зависимости от типа сообщений различают дискретные и непрерывные источники. Учитывая, что первичные сигналы часто отождествляют с передаваемыми сообщениями, в дальнейшем под источником сообщений будем понимать источник первичных сообщений разной природы и преобразователь неэлектрической величины в электрическую.

\begin{figure}[H]
	\centering
	\includegraphics[]{example-image-a}
	\caption{}
	% \label{fig:figure1}
\end{figure}
Рис. 1.1. Обобщенная структурная схема РСПИ

\begin{figure}[H]
	\centering
	\includegraphics[]{example-image-a}
	\caption{}
	% \label{fig:figure1}
\end{figure}
Рис. 1.2. Структурная схема системы передачи дискретных сообщений
Передающее устройство предназначено для преобразования сообщения x(t) в сигнал s(t), который может распространяться по линии связи. В общем случае оно выполняет операции кодирования и модуляции (рис. 1.2). При передаче непрерывных сообщений цифровыми методами передающее устройство осуществляет также операции дискретизации по времени и квантования по уровню. В узком смысле кодирование представляет собой преобразование дискретного сообщения в последовательность кодовых символов, осуществляемое по определенному правилу. В широком смысле под кодированием понимают любое преобразование сообщения в сигнал путем установления
3
взаимного соответствия. В этом последнем смысле в системах передачи непрерывных сообщений, использующих аналоговые типы модуляции, операции кодирования и модуляции совпадают.
Первичные сигналы, как правило, низкочастотные. Их можно передавать лишь по проводным линиям связи. Для передачи сообщений по радиолиниям используют специальные колебания, называемые переносчиками. Они должны хорошо распространяться по линии связи. В РСПИ в качестве переносчиков используются высокочастотные колебания.
Сами переносчики не содержат информации о передаваемом сообщении. Для того чтобы заложить в них эту информацию, применяют операцию модуляции, которая заключается в изменении одного или нескольких параметров переносчика по закону передаваемого сообщения. Устройство, осуществляющее эту операцию, называется модулятором.
В общем случае все преобразования, осуществляемые передающим устройством, можно описать с помощью некоторого оператора U, такого, что
т = итт,	(i.i)
где f(t) - колебание-переносчик.
Линия связи. Это среда, используемая для передачи сигналов. В радиолиниях средой служит часть пространства, в котором распространяются электромагнитные волны от передатчика к приемнику.
Источник помех. В реальной системе сигнал передается при наличии помех, под которыми понимаются любые случайные воздействия, накладывающиеся на сигнал и затрудняющие его прием. В общем случае действие помех n(t) можно описать с помощью оператора V, такого, что
u(t) = V[s(t), n(t)],	(1.2)
где u(t) -сигнал на входе приемника.
В частном случае
u(t) = s(t) + n(t),	(1.3)
где n(t) не зависит от s(t). Помеха, удовлетворяющая соотношению (1.3), называется аддитивной.
Если оператор V представляется в виде произведения u(t) = p(t)s(t), где p(t) -некоторая случайная функция, то помеха называется мультипликативной. В реальных линиях связи действует как аддитивная, так и мультипликативная помехи. При этом
и (t)= ju(t)s (t)+ n(t).
В зависимости от характера изменения во времени различают флуктуационные, импульсные (сосредоточенные во времени) и узкополосные (сосредоточенные по частоте) помехи. Флуктуационная помеха порождается различного рода флуктуациями, т.е. случайными отклонениями тех или иных физических величин от их средних значений. Так, источниками таких помех могут быть флуктуации тока в электрических цепях, обусловленные дискретной природой носителей заряда, которая проявляется в электронных лампах и полупроводниковых приборах в виде дробового эффекта; флуктуации разности потенциалов на концах любого проводника, обусловленные тепловым движением носителей заряда; воздействия радиоизлучения солнца и звезд и т. д. Флуктуационная помеха обычно представляет собой гауссовский стационарный случайный процесс с нулевым математическим ожиданием. В большинстве случаев она имеет равномерную спектральную плотность мощности в такой широкой полосе частот, что ее можно считать «белым шумом».
Импульсная помеха представляет собой случайную последовательность импульсов, следующих столь редко, что реакция приемника на текущий импульс успевает затухнуть к моменту появления очередного импульса. Типичным примером такой помехи является атмосферная помеха.
Узкополосная помеха - это помеха, спектральная плотность мощности которой
4
занимает сравнительно узкую полосу частот, существенно меньшую полосы частот сигнала. Чаще всего она обусловлена сигналами посторонних радиостанций, а также излучениями генераторов высокой частоты различного назначения (промышленных, медицинских и т. п.).
Приемное устройство. Основной задачей приемного устройства является выделение передаваемого сообщения из принятого сигнала u(t). В общем случае это достигается выполнением над принятым сигналом операций демодуляции и декодирования. Устройства, выполняющие эти операции, называются соответственно демодулятором и декодером.
Операция демодуляции заключается в преобразовании принятого модулированного сигнала, искаженного помехами, в модулирующий сигнал. В системах передачи непрерывных сообщений при аналоговой модуляции сигнал на выходе демодулятора совпадает с первичным сигналом, отображающим сообщение. Поэтому он без дальнейших преобразований поступает к получателю.
Часть приемного устройства, которая производит анализ входного сигнала и принимает решение о переданном сообщении, называется решающей схемой. В системах передачи непрерывных сообщений при аналоговой модуляции решающей схемой является демодулятор. В системах передачи дискретных сообщений с поэлементным приемом можно указать две решающие схемы: демодулятор и декодер. В системах передачи дискретных сообщений, использующих метод приема в целом, решающей схемой является устройство, осуществляющее операции демодуляции и декодирования. Действие приемника можно описать оператором W, таким, что
% ~ (t) = W\u(t) ],	(1.4)
где ~(t) - восстановленное сообщение.
Получатель сообщений - это устройство (магнитофон, ЭВМ автомат и т. п.), для которого предназначено сообщение.
Совокупность кодирующего и декодирующего устройств образует подсистему, называемую кодеком. Совокупность модулятора и демодулятора образует подсистему, называемую модемом. Заданная совокупность технических средств передачи информации, включающая среду распространения, называется каналом. Конкретный состав канала определяется кругом решаемых задач. Так, в одних случаях канал может состоять только из линии связи, в других - из модулятора, линии связи и демодулятора и т.п.
Существенным недостатком рассмотренной системы является то обстоятельство, что передающая сторона не располагает информацией о степени соответствия принятых сообщений переданным. Обеспечение двусторонней связи между источником информации и получателем позволяет устранять этот недостаток. Для двустороннего обмена информацией помимо прямого канала необходим второй, обратный канал (рис. 1.1). При этом информация, передаваемая по обратному каналу, может быть использована для увеличения достоверности передачи сообщений в прямом направлении.
Системы связи, в которых применяется передача информации по обратному каналу для повышения достоверности передачи по прямому каналу, называются системами с обратной связью. В зависимости от характера передаваемой по обратному каналу информации и от способа ее использования различают системы с управляющей и с информационной обратной связью. В системах первого типа решающая схема приемника либо выносит решение о переданном сообщении и направляет его получателю, либо, если это сообщение оказывается сомнительным, принимает решение повторить его, о чем передающая сторона информируется по обратному каналу. В системах второго типа приемная сторона информирует передающую по обратному каналу о том, какое сообщение им принято. Для этого используется либо ретрансляция восстановленного сообщения, либо передача некоторого сигнала, сформированного по определенному закону из принятого. Передатчик сравнивает принятое по обратному каналу сообщение с
5
переданным и при их несоответствии повторяет переданное сообщение.
В некоторых системах по обратному каналу передаются испытательные сигналы, с помощью которых определяются промежутки времени «хорошего» состояния прямого канала (например, промежутки времени, когда ослабление сигнала не превышает некоторого фиксированного значения). Именно в эти промежутки времени ведется передача информации по прямому каналу. Такие системы называются системами с прерывистой связью. По обратному каналу могут также передаваться команды на смену рабочей частоты, изменение скорости передачи информации, смену кода и т.п., что, например, имеет место в адаптивных СПИ.
По одной линии можно обеспечить одновременную передачу нескольких сообщений. Такие системы связи называются многоканальными. Для разделения канальных сигналов необходимо, чтобы они различались между собой по некоторому признаку. На практике широко применяют многоканальные системы с разделением сигналов по времени, частоте и форме.

\section{Cигналы с угловой модуляцией}
Весьма часто в качестве колебания-переносчика используется гармоническое колебание высокой частоты. В процессе модуляции в несущем колебании f(t) = Uocost+ф) можно изменять амплитуду (амплитудная модуляция), а также либо частоту ш (частотная модуляция) либо фазу ф (фазовая модуляция), оставляя амплитуду постоянной. Поскольку аргумент yt+ф является полной фазой и определяет текущее значение фазового угла, такие сигналы получили название сигналов с угловой модуляцией.
Если полная фаза процесса T(t) = ш0у + kX(t), где X(t) - сообщение, к -
коэффициент	пропорциональности,	шо -
значение частоты в отсутствии сообщения X(t), то имеем сигнал с фазовой модуляцией
Уфм(к)=иocos[oot+kx(t)].	(2.1)
\begin{figure}[H]
	\centering
	\includegraphics[]{example-image-a}
	\caption{}
	% \label{fig:figure1}
\end{figure}
Если сообщение X(t)=const, то ФМ - сигнал является простым высокочастотным сигналом.
Если X(t)=AcosQt, то с увеличением значений сообщения X(t) полная фаза y(t) растет во времени быстрее, чем по линейному закону. При уменьшении значений модулирующего сообщения происходит спад скорости роста y(t) во времени (см. рис. 2.1).
В моменты времени, когда сигнал X(t) достигает экстремальных значений, абсолютный угол между ФМ - сигналом и смодулированным гармоническим колебанием оказывается наибольшим.
\begin{figure}[H]
	\centering
	\includegraphics[]{example-image-a}
	\caption{}
	% \label{fig:figure1}
\end{figure}
\subsection{ Фазовая модуляция (ФМ)}
Предельное значение этого фазового сдвига называют девиацией фазы Ау. В общем случае, когда сообщение X(t) изменяет знак, принято различать девиацию фазы вверх Aye=kXmax и девиацию фазы вниз Xmin.
6
\subsection{Частотная модуляция (ЧМ)}
Мгновенная частота a(t) сигнала с угловой модуляцией определяется как первая производная от полной фазы по времени, т.е. мгновенная частота - это скорость изменения полной фазы:
% / \ d/
a(t ) =---мгновенная частота.
dt
(2.2)
Откуда, полная фаза равна:
/(t )= Ja(r)dr + 90,
(2.3)
о
где ф0 - начальная фаза в момент времени t=0.
При ЧМ - сигнале между сообщением X(t) и мгновенной частотой a(t) будет связь
вида
Поэтому из (2.3) и (2.4)
a(t ) = а0 + kX(t).
t
|(r)dr +фо , .0 
Kt (t) = U 0 cos /(t) = U 0 cos
(2.4)
Kt (t)= U0cos
a0t + k IX (r)dz + ф0
0
(2.5)
В соответствии с (2.4) параметрами ЧМ - сигнала являются девиация частоты вверх Aae=kXmax и девиация частоты вниз AaH=kXmin
Если X(t) - достаточно гладкая функция, то внешне осциллограммы ФМ и ЧМ - сигналов не отличаются (рис.2.2). Однако имеет место принципиальная разница: фазовый сдвиг между ФМ - сигналом и
смодулированным пропорционален X(t), для ЧМ этот сдвиг пропорционален интегралу от X(t) (сравните (2.1) и (2.5)). Т.е. ЧМ и ФМ - сигналы ведут себя по-разному при изменении частоты модуляции и амплитуды модулирующего колебания.
При ЧМ девиация частоты Аа ~ A (амплитуде НЧ - сигнала), в то же время девиация частоты Аа не зависит от частоты Q модулирующего сигнала.
\begin{figure}[H]
	\centering
	\includegraphics[]{example-image-a}
	\caption{}
	% \label{fig:figure1}
\end{figure}
При ФМ индекс модуляции m = —— A - амплитуде НЧ - сигнала независимо от
частоты модуляции. Как следствие этого, девиация частоты при фазовой модуляции линейно увеличивается с ростом частоты модулирующего сигнала (рис. 2.3).
7
Таким образом, при гармоническом модулирующем сигнале различие между ЧМ и ФМ можно выявить, только изменяя частоту модуляции.
\subsection{Общие соображения о спектре сигналов с угловой модуляцией}
Если колебание
V(t)=U0cos[шt+ф(t)]	(2.6)
получено с помощью ФМ, то ф(0 и X(t) полностью совпадают по форме и отличаются лишь постоянными коэффициентами. При этом очевидно, с точностью до постоянного коэффициента совпадают спектры функций ф(0 и X(t).
При ЧМ функция ф(0 является интегралом от передаваемого сообщения X(t). Т.к. интегрирование является линейным преобразованием, то при ЧМ спектр функции ф(0 состоит из тех же компонент, что и спектр сообщения X(t), но с измененными амплитудами и фазами.
Отвлекаясь от способа осуществления угловой модуляции и считая заданным спектр функции ф(0, находим спектр модулированного колебания V(t). Для этого выражение (2.6) преобразуем к виду:
V (t) = U 0 cos p(t )cos o0t - U 0 sin p(t )sin co0t = uc (t)-us (t)	(2.7)
Из этого выражения следует, что модулированное по углу колебание можно рассматривать как сумму двух квадратурных колебаний: косинусного uc и синусного us, каждое из которых модулировано только по амплитуде. Закон AM для косинусного колебания определяется медленной функцией cos9(t), для синусного - функцией sin9(t). Но для определения спектра AM колебания достаточно сдвинуть на частоту оо спектр огибающей амплитуд. Следовательно, для нахождения спектра колебания u(t), определяемого выражением (2.7), необходимо найти сначала спектры функций cos9(t) и sin9(t), т.е. спектры огибающих квадратурных колебаний.
Из приведенных рассуждений следует, что при одном и том же передаваемом сообщении спектр колебания, модулированного по углу, значительно сложнее, чем спектр модулированного по амплитуде. Действительно, т.к. cos9(t) и sin9(t) являются нелинейными функциями своего аргумента ф(, то спектры этих колебаний могут существенно отличаться от спектра функции ф(Д
Это обстоятельство, а также наличие двух квадратурных слагаемых, показывает, что при угловой модуляции спектр модулированного колебания нельзя получить простым сдвигом спектра колебания на величину несущей частоты оо, как это имеет место при AM.

\subsection{Спектральное разложение ЧМ и ФМ сигналов при малых индексах модуляции} % (fold)
	
Задачу о представлении сигналов с угловой модуляцией посредством суммы гармонических колебаний несложно решить в том случае, когда rn<<1. Для тонально-модулированного колебания	)=и0[ш<+ф)], т.к.	ф(t)=msшQt,	имеем
V(t)=Uocos[oot+rnsinfit].
Для этого преобразуем эту формулу следующим образом:

\begin{figure}[H]
	\centering
	\includegraphics[]{example-image-a}
	\caption{2.4 рис}
	% \label{fig:figure1}
\end{figure}
Таким образом, при m<<1 в спектре сигнала с угловой модуляцией содержатся несущая и верхняя и нижняя боковые компоненты. Индекс m играет здесь такую же роль, как и в AM - сигнале.
Однако колебание нижних боковых частот имеет сдвиг по фазе 180°. При увеличении фазового отклонения, т.е. при возрастании т, уравнение (2.8) и спектр не дают правильного представления
о действительной картине явлений при ЧМ и ФМ. Это объясняется тем, что с помощью колебаний несущей частоты и всего лишь одной пары боковых частот невозможно представить колебание, частота которого или фаза изменяется в широких пределах, а амплитуда остается строго постоянной. Для получения правильной картины необходимо учитывать боковые частоты высших порядков.

\subsection{Спектр сигнала с угловой модуляцией при произвольном значении индекса модуляции}
Итак, при тональной угловой модуляции
U(t) = U0 cos(©0t + тsin Qt) = U0 Re(e j©0t l, ejmsmQ).	(2.9)
В теории бесселевых функций доказывается, что экспонента exp{+jmsinZ}, периодическая на отрезке -tc<Z<tc разлагается в периодический ряд Фурье
ад
ejmsinZ = £ jk (mykZ ,	(2.10)
к=-ад
где m - любое вещественное число, Jk(m) - функция Бесселя к - порядка от аргумента m (рис.2.5).
ф Shm
q0-Q
U0
mUn/2
0 | ©
-mU0/2
©0+Q

\begin{figure}[H]
	\centering
	\includegraphics[]{example-image-a}
	\caption{2.5 рис}
	% \label{fig:figure1}
\end{figure}
Сравнивая (2.9) и (2.10) и подставляя Z=Qt, перепишем (2.9) так:
U (t ) = U 0Re(
ejm0t e'jmsin Qt 1
)= U 0 Re
,j0t
ад
Z Jk(m)e
jkQt
(2.11)
V	к=-ад	J
Отсюда получаем следующую модель ЧМ-ФМ - сигнала с любым значением индекса модуляции
(2.11’)
ад
U (t )= U 0 £ Jk (m)cos(©0 + kQ)t
к=-ад
Напомним, что при m<<1 ширина спектра ЧМ как и у AM равна 2Q. При значении m в пределах от 0,5 до 1 приобретает некоторое значение вторая пара боковых частот, ввиду
9
чего ширина спектра приблизительно равна 4Q. Далее при 1<m<2 приходится учитывать третью и четвертую пары боковых частот и т.д.
Vk/U0
1,0 -0,8 -0,6- -0,4-0,2
0
k=0
k=
k=-3k=-2
m=1
1

-»0
Юо
3
<<Jk/U0
1,0 -	U0 - амплитуда несущего
0,8 -	колебания
0,6--
0,4--
0,2--
0
k=-1	k=1
k=-3k=-2	k=0	k I2 k=3 .
		
m=2	Юо	

\begin{figure}[H]
	\centering
	\includegraphics[]{example-image-a}
	\caption{2.6 рис}
	% \label{fig:figure1}
\end{figure}
Фазы колебаний на рисунке 2.6 не учитываются, однако следует иметь в виду что, при нечетных k амплитуды нижних боковых следует брать со знаком минус. Дело в том, что в теории Бесселевых функций доказано, что функции с положительными и отрицательными индексами
J k (m) = (-1)kjk (m),
поэтому начальные фазы боковых колебаний с частотами юо+kQ и юо-kQ совпадают, если k - четное, и отличаются на 180°, если k - нечетное.
Чем больше индекс k функции Бесселя, тем протяженнее область аргументов, при которых эта функция мала. Важно отметить, что с ростом индекса модуляции m расширяется полоса частот, занимаемая сигналом. Обычно полагают, что допустимо пренебречь всеми спектральными составляющими с номером lkl>m+1. Отсюда следует оценка практической ширины спектра с угловой модуляцией nnp=2kQ или nnp=2(m+1)Q.
Как видно, реальные ЧМ и ФМ - сигналы характеризуются условием m>> 1, итак:
nnp~2mQ=2A.	(2.12)
Таким образом, сигнал с угловой модуляцией занимает полосу, приблизительно равную удвоенной девиации частоты.
Отметим, что для передачи AM - сигнала требуется полоса в m раз меньшая. Большая полоса, занимаемая ЧМ и ФМ, обусловила их большую помехозащищенность, однако в целях радиосвязи их применение целесообразно в диапазоне УКВ (метровых и более коротких волн).
Вопрос: Возможно ли отсутствие в спектре ЧМ несущего колебания на частоте ю0?
Ответ: Да, т.к. при перераспределении мощность Vm по частотным составляющим в соответствии с Jn(m) может быть j0(m)=0 при m = 2,405; 5,520; 8,654; 11,792 и т.д.


\section{Частотная модуляция в автогенераторе}
Теперь рассмотрим основные принципы осуществления модуляции, т.е. изменение радиочастотного колебания по закону управляющего сигнала.
Задачу получения ЧМ колебаний можно сформулировать как задачу создания генератора гармонических колебаний, частота которого должна изменяться в соответствии с законом изменения управляющего сигнала. Частота колебаний генератора
1
определяется резонансной частотой контура ю0 = .--- и, следовательно, для ее
изменения необходимо менять либо емкость C, либо индуктивность L.
		da0	ю0	Аю	1 AC
Продифференцировав ю0, например, по C получим —- = -— или -----«-------.
dC	2C	ю0	2 С0
Аналогично
10
(3.1)
Аа	1 AL
Щ>	2 L0
Как видно, при малых изменениях частоты можно считать, что она пропорциональна емкости, т.е., желая получить модулированное колебание, следует изменить емкость (или индуктивность) контура в соответствии с передаваемым сообщением. Итак, контур должен содержать емкостной (или индуктивный) параметрический элемент.
Широко распространенным способом электронного управления является подключение к контуру варикапа, емкость которого зависит от напряжения, приложенного в направлении запирания перехода.
Упрощенная схема автогенератора с варикапом изображена на рис. 3.2.
\begin{figure}[H]
	\centering
	\includegraphics[]{example-image-a}
	\caption{3.1 рис}
	% \label{fig:figure1}
\end{figure}
\begin{figure}[H]
	\centering
	\includegraphics[]{example-image-a}
	\caption{3.2 рис}
	% \label{fig:figure1}
\end{figure}
Разделительный конденсатор Ср, предназначен для развязки контура от источника E0. Источник Е0 устанавливает начальную рабочую точку (НРТ) на вольт-фарадной характеристике варикапа. Блокировочный дроссель Ldp предназначен для того, чтобы ВЧ ток от автогенератора не проходил в источник ЭДС en(t).
На схеме замещения (рис. 3.2-6): С0 - средняя емкость в отсутствии модулирующего колебания, AC(t) - вариация емкости в зависимости от eo(t). Сопротивление p-n перехода R, объемное сопротивление полупроводника г.
Если напряжение на емкости достаточно мало, то, как отмечено выше, нелинейный элемент (в данном случае емкость) можно трактовать как линейный параметрический. Принимая рабочий участок «а-а» зависимости C(U) (рис. 3.1) за прямую линию получим следующее.
Если управляющее напряжение меняется по закону
uynp=E0 +UccosQt,
то емкость меняется по закону
C=Co(1+rncCOsQt),
11
где тС=АС/С0. Соответственно частота o0(t)« ш0Н - Аш cosQt, где ш0H =
л	АС	1
Аш « ш0H	= 2 mC0H - ДевиаЦия частоты
1
LCn
(3.2)
\section{Демодуляция сигналов}
Детектирование колебаний заключается в выделении сообщения, которое в неявной форме содержится в модулированном ВЧ колебании. Детектирование является процессом обратным процессу модуляции. Соответственно основным видам модуляции различают амплитудное, частотное и фазовое детектирование.
При частотной модуляции, как известно, полезное сообщение пропорционально отклонению мгновенной частоты сигнала от частоты несущего колебания.
При ЧМ детектирование обычно осуществляется в результате преобразования ЧМ  AM или ЧМ  ФМ и последующего амплитудного или фазового детектирования.
Рассмотрим для примера частотное детектирование методом преобразования ЧМ 
AM.
Учтем, что разложение АЧХ полосового фильтра имеет вид:
f
% \K(jo) = \K(j&o)| + КОЧ)| Ц?)-Шо)+ 	(4.1)
Тогда можно ЧМ превратить в неглубокую AM, подавая детектируемый сигнал на линейный частотный фильтр, настроенный таким образом, чтобы в разложении АЧХ
f
коэффициент К(jo0) (крутизна АЧХ в исходной точке) был максимален (середина линейного участка). Этот способ показан на рис. 4.1. Uag(t) = E0cos[40 + A4COS Qt]. Полагая, что ш(г) = ш0 +A4COS Qt, получим на выходе фильтра сигнал со сложной амплитудно-угловой модуляцией. Мгновенная амплитуда переменной составляющей
f
этого сигнала изменяется во времени по закону Vmg(t) = b0 |К(jo) Аа cos Qt, где b0 -постоянный коэффициент. Окончательная обработка проводится обычным AM детектором, включаемым на выходе фильтра |К(jo).
Этот метод имеет недостаток, обусловленный малым диапазоном линейности характеристики детектирования (скат частотной характеристики фильтра) и необходимостью настройки на частоту, отличную от частоты немодулированного колебания (шрезш0).
\begin{figure}[H]
	\centering
	\includegraphics[]{example-image-a}
	\caption{4.1 рис}
	% \label{fig:figure1}
\end{figure}
Для устранения этого недостатка совмещают в одной схеме два контура и два амплитудных детектора (рис. 4.2).
Здесь контур 1 настроен на частоту несколько большую, чем ш0, а контур 2, наоборот, на частоту немного меньшую, чем ш0 (рис. 4.3-а).
\begin{figure}[H]
	\centering
	\includegraphics[]{example-image-a}
	\caption{4.2 рис}
	% \label{fig:figure1}
\end{figure}
Выпрямленные напряжения ивых1 и ивых2 пропорциональны (при неискаженном AM детектировании) амплитудам U1 и U2, но имеют противоположные полярности (рис. 4.3-6). Поэтому выходное напряжение Ueblx=Ueblx1-Uвых2. Результирующая детекторная характеристика Ueblx=Ueblx) приведена на рис. 4.3-е.
\begin{figure}[H]
	\centering
	\includegraphics[]{example-image-a}
	\caption{4.3 рис}
	% \label{fig:figure1}
\end{figure}
Как видно из рис. 4.3-е эта характеристика имеет достаточно протяженный участок а-а, близкий к прямой линии.

\section{RC--генератор низкочастотных гармонических колебаний}
Автогенераторы с LC-контурами, рассмотренные в разделе 3, обычно применяются на достаточно высоких частотах (десятки килогерц и выше). В лабораторном макете, используемом в данной лабораторной работе, частота колебания - переносчика выбрана в звуковом диапазоне частот (от 10 до 20 кГц). На таких низких частотах автогенераторы на LC-контурах обычно не применяются, т.к. необходимые значения индуктивностей катушек очень велики и индуктивности получаются громоздкими. Поэтому на частотах от нескольких герц до нескольких десятков килогерц обычно используют RC-автогенераторы гармонических колебаний, представляющие собой комбинацию активных нелинейных усилителей и RC-цепей в качестве элементов обратной связи. Именно такой автогенератор
14
с управляемой частотой генерации реализован в лабораторном макете в составе частотного модулятора.
Рассмотрим уравнение автогенератора с LC-контуром:
d 2U f R
hF Л L
MMUl 1 U+<U=о
LC dU ) dt p
(5.1)
Это уравнение имеет второй порядок. Действием обратной связи, реализованной на трансформаторной связи с взаимной индуктивностью М, коэффициент при первой R M dq>(U)
L LC dU
производной
обращается в нуль и отрицательную величину в
стационарном режиме работы автогенератора и в режиме его самовозбуждения соответственно. Следовательно, для осуществления RC-автогенератора нужно так составить схему, чтобы она описывалась таким же дифференциальным уравнением второго порядка, как и (5.1).
Рассмотрим схему, представленную на рис. 5.1.
U2
\begin{figure}[H]
	\centering
	\includegraphics[]{example-image-a}
	\caption{5.1 рис}
	% \label{fig:figure1}
\end{figure}
Дифференциальное уравнение этой схемы
dU	t
dt
+ 2aU2 +0 f U2dt = Uy или *	/Г12
d U2 „ U2	2tj	1 dU,
—+ 2 — + alU 2 =-------------1
dt	dt	t12 dt
(5.2)
где a02 = 1/R1R2C1C2;	2a	= 1/t1	+ 1/t2 + 1/t12;	t1	= R1C1;	t12	= R1C2;	t2	= R2C2.
Левая часть уравнения (5.2) совершенно аналогична левой части уравнения (5.1). Если теперь в эту схему внести нелинейный усилитель и охватить всю схему обратной связью, как показано на рис.	5.2, то полученная схема сможет генерировать
гармонические колебания на одной из частот, определяемых уравнением (5.2)
\begin{figure}[H]
	\centering
	\includegraphics[]{example-image-a}
	\caption{5.2 рис}
	% \label{fig:figure1}
\end{figure}
Найдем дифференциальное уравнение схемы рис. 5.2, положив в ней K()=U1/U2, где U1 -выходное напряжение усилителя К(о), a U2 - входное. Из этого уравнения можно
% >> ТТ	\ТГ dU 1	ът dU2
наити U1=K(q)U2; —— = K
dt
dt
d2U2 + 0 „ 2
--- + 2a —
dt	dt
f
Подставляя полученное соотношение в (5.2), получим
U K dU
 21
dt
2 + a>lU 2 = 0 или
d 2U
dt
2
+
1	1	1 - K
— + — +-----
Vr1
 + lU 2 = 0,
dt
(5.3)
'2	l,21 J
где i21 = R2C1.
Уравнение (5.3) совершенно аналогично уравнению (5.1) с LC-контуром. Для осуществления самовозбуждения необходимо сделать отрицательным коэффициент при первой производной
(
1
+ -
1
+ -
1 - K

< 0.
(5.4)
R1C1 R2C2 R2C1
Типичная схема RC-автогенератора приведена на рис. 5.3 (а), а его эквивалентная схема с разомкнутой цепью обратной связи - на рис. 5.3 (б). Здесь К0 - идеальный усилитель с вещественным и положительным коэффициентом усиления К0. Выход усилителя соединяется с его входом через пассивный четырехполюсник, выделенный на рисунке пунктирной рамкой и представляющий собой цепь положительной обратной связи. Передаточная функция этой цепи равна
R2 / jaC2 R2 +1/ jaC2
P( j) =
(r1 +1/jc1)+ Rz/ J0)Cl-
V1	1 R2 +1/jC2
или в переменных Лапласа
pr
21
(5.5)
АС p) = -—	,
(1 + prxX1 + РГ2) + РГ21 Характеристическое уравнение автогенератора в общем виде имеет вид
K(ja>)0(jсо) -1 = 0 или K(р)А(р) = 1	(5.6)
Для схемы рис. 5.3 с передаточной функцией цепи положительной обратной связи Р(р) в соответствии с (5.5) характеристическое уравнение имеет вид
а2 р2 + а1 р +1 = 0, где «2 = Т1Т2; «1 = Т1+Т2 - Т21(Кэ - 1).
(5.7)
16
\begin{figure}[H]
	\centering
	\includegraphics[]{example-image-a}
	\caption{5.3 рис}
	% \label{fig:figure1}
\end{figure}
Условия самовозбуждения автогенератора состоит в том, чтобы коэффициент а1 был меньше нуля. Отсюда находится условие, налагаемое на коэффициент усиления К0:
K о > 1 +
RC + RC RC
(5.8)
Частота генерации определяется мнимой частью корней характеристического уравнения (5.7). Для стационарного режима автогенератора коэффициент а1 этого уравнения равен нулю, и уравнение принимает вид
/тИ2 + 1 = 0.	(5.9)
Мнимые части корней этого уравнения и частота генерации равны
Pl,2 =±A/Vw А0 = 1
/VRiR.C,C, l,	(5'10)
Обычно выбирают R1 = R2 = R и C1 = C2 = C. При этом передаточная характеристика (5.5) принимает вид
J
Р( А) =
аг
1 + 3 Ла/ 1-1 А 71 А J 1/А
2 , где Ао = /RC 
(5.11)
уо у V/ о.
Соответствующие амплитудно-частотная и фазо-частотная характеристики представлены на рис. 5.4 (а) и (б) соответственно.
\begin{figure}[H]
	\centering
	\includegraphics[]{example-image-a}
	\caption{5.4 рис}
	% \label{fig:figure1}
\end{figure}
Условие возбуждения такого генератора на частоте ш0 при этом переходит в К0 > 3. Несмотря на то, что ЛС-генераторы существенно уступают LC-генераторам по качеству синусоидальных колебаний вследствие отсутствия высокодобротных частотноизбирательных элементов, они находят весьма широкое применение в измерительной технике.

\section{Исследование частотного модулятора}
Цель первой части лабораторной работы - исследование принципа действия частотного модулятора, получение характеристик частотного модулятора при воздействии на его вход гармонического сигнала (тональная модуляция) и исследование формы и спектра сигналов с частотной модуляцией.
\subsection{Схема работы и измерительная аппаратура}
В данной работе используется универсальный лабораторный стенд со сменным блоком ЧАСТОТНЫЙ МОДЕМ, упрощённая принципиальная схема которого приведена на рис.6.1. В первой части работы объектом исследования является левая часть схемы (между гнёздами КТ 1 и КТ 2).
\begin{figure}[H]
	\centering
	\includegraphics[]{example-image-a}
	\caption{6.1 рис}
	% \label{fig:figure1}
\end{figure}
Как видно из схемы, частотный модулятор представляет собой ЛС-генератор, состоящий их двухкаскадного резистивного усилителя (А1) и фазобалансной цепи (ФБЦ), обеспечивающей положительную обратную связь. Частота генерации зависит от параметров ФБЦ - С3, С4 и сопротивлений каналов (Rch) полевых транзисторов VT1 и VT2. Сопротивление канала (Rch) зависит от управляющего напряжения, приложенного к затвору. Таким образом, полевой транзистор в ФБЦ является параметрическим элементом, управляемым модулирующим напряжением. Напряжение смещения (Есм), являющееся постоянной составляющей модулирующего сигнала, позволяет установить несущую
18
частоту модулированного сигнала, а переменная составляющая, т.е. сам модулирующий сигнал, поданный на гнезда КТ 1, обеспечивает девиацию частоты Afmax, зависящую от амплитуды модулирующего сигнала. Выходом частотного модулятора являются гнезда КТ 2.
В схеме модулятора имеется блок автоматической регулировки усиления, поддерживающий постоянную амплитуду ЧМ-сигнала (на схеме не показан).
В качестве источника модулирующего сигнала используется встроенный диапазонный генератор НЧ с цифровой индикацией частоты выходного гармонического сигнала, подключаемый к входу модулятора. Для контроля амплитуды модулирующего сигнала используется встроенный вольтметр. Измерение частот, анализ осциллограмм и спектра сигналов производится на двухлучевом цифровом осциллографе TDS 2002.
\subsection{Лабораторное задание}
1.	Снимите статическую модуляционную характеристику и определите оптимальный режим модулятора.
2.	Определите влияние амплитуды модулирующего сигнала на форму и ширину спектра ЧМ-сигнала (при постоянной частоте модуляции).
3.	Определите влияние частоты модуляции на форму и ширину спектра ЧМ-сигнала (при постоянной амплитуде модулирующего сигнала).
4.	Наблюдайте форму сигнала на входе и выходе частотного модулятора.
6.3.	Методические указания
1.	Измерение статической модуляционной характеристики (СМХ): П=ф(Есм) выполняется при отсутствии модулирующего сигнала. Последовательно устанавливая движковым потенциометром Есм значения из таблицы 1, следует определить значения частоты модулятора f, подключив выход модулятора (гнездо КТ 2) к входу 1 канала осциллографа, работающего в режиме <<auto set>>, по цифровому индикатору в нижней части дисплея осциллографа.
Таблица 1
ш о ш	0	-0,5	-1	-1,5		-6,5
f [кГц]						
По данным таблицы строится график СМХ, на котором следует отметить:
-	положение рабочей точки (на середине линейного участка);
отсюда находят Есм опт и несущую частоту f0 (по вертикальной оси графика);
-	угол наклона линейного участка СМХ;
(тангенс этого угла соответствует коэффициенту Кчм модулятора);
-	границы линейного участка (fMIN, fMAX).
Полученные данные сведём в таблицу 2
Таблица 2
Есм опт	fo	fMIN	fMAX	Кчм
				
(В случае хорошей линейности СМХ выбор несущей частоты некритичен.)
19
2.	Исследование влияния амплитуды модулирующего сигнала на спектр ЧМ (при FMofl=const).
2.1.	По ряду заданных значений Мчм (табл. 3) следует рассчитать амплитуды модулирующих сигналов, а затем и действующие значения Uc, выбрав FMOfl = 500 Гц
Таблица 3
Мчм	0	0,1	0,5	1,0	2,4	3,8
AfMAX						
Umc						
Uc						
2Af *						
Для заполнения таблицы напомним некоторые определения для ЧМ. Индекс частотной модуляции
Мх1 =
A/MAX 1 IIA
Мгновенное значение отстройки частоты ЧМ-сигнала от f0
(1)
Af (t) = КЧм UMOfl(t)
Девиация частоты AfMAX - максимальное отклонение мгновенной частоты модулированного колебания от несущей частоты f0 или полуширина спектра ЧМ-сигнала. Такое отклонение частоты бывает при амплитудном значении модулирующего гармонического сигнала Umc
AfMAX = Кчм Umc	(2)
Отсюда можно определить необходимую для заданного значения девиации частоты модулированного колебания амплитуду модулирующего сигнала Umc= A fMAX / Кчм .
Четвёртая строка таблицы заполняется, исходя из необходимости пользоваться вольтметром переменного напряжения, имеющим градуировку в действующих значениях
Uc = Umc 0,707.
2.2. Подключить встроенный НЧ генератор (выход 0 дБ) к входу модулятора (гнездо КТ 1), соединить общие точки встроенного НЧ генератора и модулятора. К входу модулятора подключить также вольтметр переменного напряжения стенда (гнездо «+»). Общую точку модулятора соединить с отрицательным входом вольтметра. Установить частоту генератора Fmoa = 500 Гц (диапазон 2).
2.3.	Последовательно устанавливая значения UC по показаниям вольтметра из таблицы 3 регулятором амплитуды выхода НЧ генератора (ниже ручки точной регулировки частоты)), получить на осциллографе, подключённом к выходу модулятора (гнездо КТ 2) спектры ЧМ-сигналов. Для этого в экранном меню осциллографа выбрать
20
режим <<FFT>>. С помощью ручки регулировки горизонтального масштаба изображения на дисплее установить разрешение 500 Гц/дел. На каждой спектрограмме обязательно указывать:
l,	установленные значения параметров сигналов и постоянных напряжений;
l,	частоты отдельных составляющих спектра (желательно и их амплитуды);
(Для	измерения амплитуд спектральных компонент необходимо
воспользоваться кнопкой «курсор» на передней панели осциллографа, затем в экранном меню <<sours>> выбрать режим <<math>>, а в экранном меню <<type>> выбрать режим <<magnitude>>. Затем в обновленном экранном меню следует выбрать сначала курсор №1 и с помощью универсальной ручки, расположенной в левом верхнем углу панели управления осциллографа, установить его на максимум измеряемой спектральной компоненты. Так же можно использовать и курсор №2. Цифровые индикаторы на дисплее покажут значения, соответствующие положениям курсоров в децибелах относительно 1 Вольта, и разницу между ними в децибелах.)
l,	практическую ширину спектра 2Af *.
(При определении 2Af * учитывать только ту часть спектра, в которой амплитуды спектральных компонент составляют не менее (- 30) дБ от максимальных амплитуд. Для этого установить курсор №1 на уровень, соответствующий максимальной амплитуде, а курсор №2 на 30 дБ ниже).
Полученные значения 2Af * внести в таблицу 3.
Напомним, что число децибел определяет десять десятичных логарифмов отношения двух мощностей сигналов. Так, если мощность сигнала Р1 в 10 раз больше мощности сигнала Р0, выбранной за опорный уровень, то Р1 = 10 дБ относительно Р0, если Р1 в 2 раза больше Р0, то Р1 = 3 дБ относительно Р0, если Р1 в 100 раза больше Р0, то Р1 = 20 дБ относительно Р0, если Р1 в 10 раза меньше Р0, то Р1 = -10 дБ относительно Р0 и т.д. Если в качестве опорного сигнала выбран гармонический сигнал с амплитудой U0 = 1 Вольт, то [дБ/В] = 20 Lg(U1 [B]/U0 [В]), т.к. мощность сигнала пропорциональна квадрату его амплитуды.
3.	Исследование влияния частоты модуляции на спектр ЧМ-сигнала. (UC=const)
3.1.	Сохраняя схему соединений (п.2), установить значения UC из таблицы 3 для Мчм = 2,4 и не менять его в дальнейшем.
3.2.	Последовательно устанавливая частоты модуляции (табл. 4), получить спектрограммы соответствующих ЧМ-сигналов. В таблицу внести значения 2Af *. Важно! При выполнении этого задания практическую ширину спектра определять по спектральным компонентам, которые составляют не менее (-20) дБ от максимальных амплитуд. То есть компоненты спектра, мощность которых более чем в 100 раз меньше мощности максимальных (амплитуды более чем в 10 раз меньше максимальных) учитывать не следует.
Таблица 4
Uc =	 B; fo =	 кГц						
Fmoa	Гц	50	100	250	500	1000
2Af *	Гц					
Мчм						
21
3.3.	Заполнить последнюю строку табл. 4, используя определение Мчм и необходимые данные из табл. 3.
4.	Исследование формы колебаний на входе и выходе частотного модулятора.
4.1.	Соединить второй вход двухлучевого осциллографа с входом модулятора (для этого надо отключить вольтметр, сохраняя соединение с генератором). На первый вход осциллографа подать выходной сигнал модулятора.
4.2.	Установить максимальную частоту модуляции Fmoa для второго поддиапазона встроенного НЧ генератора входного сигнала (примерно 2,2 кГц), а уровень сигнала увеличивать до тех пор, пока на осциллограмме выходного сигнала не появится паразитная амплитудная модуляция. Несколько уменьшить входной сигнал так, чтобы огибающая ЧМ-сигнала снова стала почти ровной.
4.3.	Зафиксировать осциллограммы на входе и выходе частотного модулятора, нажав сначала кнопку <<auto set>>, а затем кнопку <<run/stop>> на передней панели осциллографа. Оценить частоты выходного модулированного колебания на интервалах времени, соответствующих максимальному и минимальному уровням модулирующего колебания. Для этого следует использовать курсоры времени. Для отображения курсоров времени на дисплее осциллографа необходимо нажать кнопку «курсор» на передней панели осциллографа. В экранном меню <<source>> выбрать <<Ch 1>>, в экранном меню <<type>> выбрать <<time>>. Выбирая поочередно первый и второй курсоры, совместить их с соседними максимумами (минимумами) осциллограммы выходного сигнала в окрестности минимума, а затем максимума входного сигнала. Зафиксировать показания цифрового индикатора << 1/At>> для этих двух временных интервалов.
5.	Требования к отчету по первой части работы
Отчет должен содержать:
1.	Схему частотного модулятора.
2.	Статическую модуляционную характеристику.
3.	Спектры, таблицы и осциллограммы по всем пунктам исследований.
4.	Теоретический расчёт спектров для
l,	п.2.1., для Мчм = 2,4 (из табл. 3)
l, п.3.1., для Fmoa= 250 Гц (из табл. 4)
5.	Для расчётов принять Um0=1B (амплитуда немодулированного сигнала)
6.	Выводы по пунктам 2 и 3.

\section{Исследование детектора ЧМ-сигналов}
Цель второй части лабораторной работы - экспериментальное исследование частотного детектора и выбор оптимального режима детектирования
7.1.	Схема работы и используемая аппаратура
Во второй части лабораторной работы используется тот же универсальный лабораторный стенд со сменным блоком ЧАСТОТНЫЙ МОДЕМ (рис. 6.1). В качестве источника ЧМ-сигнала используется частотный модулятор, рассмотренный в первой части работы. Выход частотного модулятора (гнездо КТ 2 на рис. 6.1) соединяется перемычкой с входом частотного детектора (гнездо КТ 3). Выход частотного детектора (гнездо КТ 4) постоянно соединен с микроамперметром, измеряющим постоянную составляющую тока детектора (см. рис. 6.1). Схема частотного детектора состоит из усилителя на полевом транзисторе VT3, в нагрузку которого включены два резонансных
22
контура, настроенных на разные частоты (foi и fo2). Эти частоты расположены симметрично относительно несущей частоты ЧМ-сигнала. Ток ЧМ-сигнала с постоянной амплитудой, протекая через два расстроенных контура, вызывает на них падения напряжения, пропорциональные их сопротивлениям. Чем ближе мгновенная частота ЧМ-сигнала к резонансной частоте контура, тем больше амплитуда напряжения на контуре и наоборот. Таким образом, линейная цепь (рассмотренный колебательный контур) преобразует ЧМ-сигнал в сигнал, в котором и амплитуда, и частота меняются одновременно. Осциллограмма такого сигнала внешне очень похожа на AM-сигнал, но частота заполнения его меняется так же, как у входного ЧМ-сигнала. Нагрузкой каждого контура является свой детектор огибающей (АМ-детектор).
Выходные напряжения AM-детекторов (на резисторах R5 и R6) зависят от расстройки контуров относительно мгновенной частоты ЧМ-сигнала. Для идеальной работы ЧМ-детектора модуль полного сопротивления расстроенного контура должен меняться прямо пропорционально девиации частоты ЧМ-сигнала. Однако на частотной характеристике контура имеется небольшой почти линейный участок в районе точки перегиба. Для увеличения ширины линейного участка характеристики детектирования применяют не один, а два симметрично расстроенных контура. Встречное включение диода (VD2) во втором детекторе огибающей позволяет в значительной степени компенсировать нелинейность склона АЧХ контура, а также компенсировать постоянную составляющую выходного сигнала.
Выходное напряжение ЧМ-детектора (гнездо КТ 3) равно разности напряжений на выходах AM-детекторов: Ubmx = Ur5 - UR6.
В работе также используются встроенные диапазонный НЧ генератор, приборы постоянного и переменного напряжений, двухлучевой цифровой осциллограф.
7.2.	Лабораторное задание
1.	Снимите характеристику детектирования и выберите оптимальный режим работы частотного детектора.
2.	Наблюдайте сигналы на входе модулятора и выходе детектора в оптимальном режиме и при отклонениях от него.
7.3.	Методические указания
1.	Снятие характеристики детектирования	I0 = 92(f) производится при отсутствии
модулирующего сигнала (Мчм=0) путём изменения частоты входного сигнала с измерением постоянной составляющей тока детектора. При этом вход модулятора отключен (гнездо КТ 1 свободно), между гнездами КТ 2 и КТ 3 установлена перемычка, а управление частотой осуществляется изменением смещения (Есм) в модуляторе. Измерение частоты на выходе модулятора (гнездо КТ 2) производится с помощью цифрового осциллографа аналогично пункту 1 раздела 6.3, а ток I0	-
микроамперметром, расположенным над регулятором Есм (множитель х1). Очевидно, что если пункт 1 раздела 6.3 уже выполнен, необходимость в измерении частот отпадает, и первые две строчки из таблицы 1 переносятся в таблицу 5.
Таблица 5
ш о ш	0	-0,5	-1	-1,5		-6,5
f [Кгц]						
!о [мкА]						
23
1.1.	Изменяя напряжение смещения (Есм) в соответствии с таблицей 5 измерить частоты выходного сигнала модулятора и, одновременно с этим, ток детектора I0. При заполнении таблицы 5 кроме указанных значений Есм следует добавить те значения Есм и f, при которых I0 принимает нулевое и экстремальные значения.
1.2.	По результатам таблицы 5 строится статическая модуляционная характеристика (СМХ) и характеристика детектирования (ХД). Из графика ХД следует определить оптимальное значение несущей частоты f0, соответствующее нулевому току детектора и максимальную для данного детектора девиацию частоты A fMAX, соответствующую границе линейного участка ХД, считая от частоты f0. Из СМХ определяют напряжение смещения Есм опт, при котором несущая частота равна f0 и максимальную амплитуду сигнала UMc, при которой девиация частоты составит AfMAX. Полученные значения параметров внести в таблицу 6.
Таблица 6
Есм опт ,В	fo , кГц	AfMAX , кГц	Umc max ,B	Uc MAX ,B
				
Соединить выход встроенного НЧ генератора с гнездом КТ 1 (вход модулятора). Туда же подключить вольтметр переменного напряжения. Установить на генераторе гармонический сигнал с частотой Fmoa=200 Ец и действующим значением UC мах (по вольтметру). Установить Есм = Есм опт. (из табл.6). Отключив вольтметр от КТ 1, подключить к ней первый вход двухлучевого осциллографа, а второй его вход соединить с выходом детектора (гнездо КТ 4).
1.3.	Получив неподвижные осциллограммы, зафиксировать их в отчёте. Обратить внимание на «зубцы» выходной осциллограммы, связанные с работой амплитудных детекторов. Для этого целесообразно воспользоваться кнопкой <<run/stop>>. Сравнить форму осциллограмм. Для этого целесообразно инвертировать осциллограмму выходного сигнала, воспользовавшись кнопкой <<Ch 2>>, и в появившемся экранном меню, выбрать режим <<invert on>>. С помощью ручек регулировки вертикального смещения лучей совместить их на дисплее.
2.	Работа детектора в неоптимальном режиме возникает при выходе сигнала за пределы линейного участка ХД.
2.1.	Изменить напряжение смещения на +1 В от оптимального. По графику СМХ или таблице 5 определить новое значение несущей частоты и внести его в отчёт. Повторить п.1.3.
Повторить п.2.1, но при ЕСм = (Есм опт - 1) В.
Восстановить прежнее значение Есм опт. Сначала уменьшить до нуля, а затем плавно увеличивать амплитуду модулирующего сигнала UmC до появления заметных искажений выходного сигнала.
Повторить п.1.3.
3.	Требования к отчету по второй части работы
Отчет должен содержать:
1.	Принципиальную схему частотного детектора.
2.	Статическую модуляционную характеристику частотного модулятора.
3.	Характеристику детектирования.
4.	Временные диаграммы оптимального и неоптимальных режимов.
24

\section{Контрольные вопросы}
1.	Что такое угловая модуляция?
2.	Дайте определение ЧМ-сигнала.
3.	Назовите известные Вам области применения ЧМ сигналов.
4.	Запишите аналитическое выражение сигнала с тональной частотной модуляцией. Приведите пример записи тонального ЧМ-колебания с параметрами fo = 100 МГц; Бмод = 10 КГц; AfMAX = 50 КГц, определите значение индекса угловой модуляции.
5.	Как рассчитать спектр ЧМ-колебания?
6.	Представьте (качественно) спектр колебания i (t) = I m0 cos (ю0 t + 0,01 cos Q t).
7.	Какое отношение имеют функции Бесселя к спектру ЧМ-сигнала?
8.	Сколько спектральных линий надо учесть в практической ширине спектра ЧМ при Мчм = 4?
9.	Амплитуда несущего колебания и амплитуда немодулированного ЧМ-сигнала - это одно и то же или нет?
10.	Тот же вопрос для АМ-сигнала.
11.	Опишите принцип действия частотного модулятора. Какие способы получения ЧМ-колебаний Вам известны?
12.	Статическая модуляционная характеристика и её смысл.
13.	Что такое МЧм и КЧм?
14.	Какие требования предъявляются к частотному модулятору и частотному детектору?
15.	Какие функции выполняют частотный модулятор и частотный детектор?
16.	Как работает частотный детектор?
17.	Что такое оптимальный режим частотного детектора?
\section{Литература}
1.	Радиотехнические системы передачи информации / Под ред. В.В. Калмыкова. -М.:Радио и связь, 1990. - 304 с.
2.	Орлов ИЯ. Курс лекций по основам радиоэлектроники. Н.Новгород: ИНГУ, 2005. -
3.	Астайкин А.И., Помазков А.П. Теоретические основы радиотехники. Ч. III. Сигналы в радиотехнических цепях. Саров: ФГУП РФЯЦ-ВНИИЭФ, 2004. - 400 с.
4.	Борисенко А.В. Теория электрической связи. Методические рекомендации к лабораторным работам. Ч. I. Санкт-Петербург: СПбГУТ; СПбГУИТМО, 2004. - 64 с.
168 с.
\end{document}
